\documentclass[a4paper,twoside]{article}

\usepackage{epsfig}
\usepackage{subfigure}
\usepackage{calc}
\usepackage{amssymb}
\usepackage{amstext}
\usepackage{amsmath}
\usepackage{amsthm}
\usepackage{multicol}
\usepackage{pslatex}
\usepackage{apalike}

%\usepackage[utf8]{inputenc}
\usepackage[fleqn]{amsmath}
\usepackage{amssymb}

\usepackage{tikz}
\usetikzlibrary{backgrounds}
\usetikzlibrary{arrows}

\usepackage{algorithm}
\usepackage{algpseudocode}
\usepackage{algorithmicx}

\usepackage{marvosym}
\usepackage{url}

\usepackage{SCITEPRESS}     % Please add other packages that you may need BEFORE the SCITEPRESS.sty package.

\renewcommand{\.}{. \;}
\theoremstyle{definition}
\newtheorem*{definition}{Definition}
\newtheorem*{example}{Example}

\theoremstyle{plain}
\newtheorem{theorem}{Theorem}

\theoremstyle{remark}
\newtheorem*{remark}{Remark}

\newcommand{\pto}{\nrightarrow}
\newcommand{\ptree}[3]{
	\ifx&#2&%
	(#1 \rightrightarrows #3)
	\else
	(#1^#2 \rightrightarrows #3)
	\fi
}

\newcommand{\allgraphs}[1]{\mathcal{G}_{#1}}
\newcommand{\alltgraphs}[1]{\mathcal{TG}_{#1}}
\newcommand{\emptyGraph}{\varepsilon}
\newcommand{\startG}[1]{Z_{#1}}
\newcommand{\startTG}[1]{Z_{#1}}
\newcommand{\pro}{\to}

\newcommand{\rel}{\sim}
\newcommand{\rrestr}{\triangleright}

\newcommand{\source}{\operatorname{s}}
\newcommand{\Source}{\operatorname{S}}
\newcommand{\target}{\operatorname{t}}
\newcommand{\Target}{\operatorname{T}}


\newcommand{\neigh}[1]{\operatorname{neigh}_{#1}}
\newcommand{\scont}[2]{\eta_{#1}(#2)}
\newcommand{\cont}[2]{\operatorname{cont}_{#1}(#2)}
\newcommand{\emb}[1]{\operatorname{emb}(#1)}

\newcommand{\st}{\; | \;}
\newcommand{\isomorph}{\cong}
\newcommand{\ms}[1]{\overset{#1}{\leftarrow}}
\newcommand{\mt}[1]{\overset{#1}{\rightarrow}}
\newcommand{\cderiv}[3]{
	\ifx&#2&%
	\overset{#1}{\Rrightarrow}_{#3}
	\else
	\overset{#1,#2}{\Rrightarrow}_{#3}
	\fi
}
\newcommand{\deriv}[3]{
	\ifx&#2&%
	\overset{#1}{\Rightarrow}_{#3}
	\else
	\overset{#1,#2}{\Rightarrow}_{#3}
	\fi
}
\newcommand{\derivtr}[3]{
	\ifx&#2&%
	\overset{#1}{\Rightarrow^*}_{#3}
	\else
	\overset{#1,#2}{\Rightarrow^*}_{#3}
	\fi
}
\newcommand{\tcderiv}[3]{
	\ifx&#2&%
	\overset{#1}{\Rrightarrow}_{#3}
	\else
	\overset{#1,#2}{\Rrightarrow}_{#3}
	\fi
}
\newcommand{\tderiv}[3]{
	\ifx&#2&%
	\overset{#1}{\Rightarrow}_{#3}
	\else
	\overset{#1,#2}{\Rightarrow}_{#3}
	\fi
}
\newcommand{\tderivtr}[3]{
	\ifx&#2&%
	\overset{#1}{\Rightarrow^*}_{#3}
	\else
	\overset{#1,#2}{\Rightarrow^*}_{#3}
	\fi
}

\newcommand{\?}{\operatorname{\textbf{?}}}
\renewcommand{\:}{\operatorname{\textbf{:}}}
\newcommand{\Break}{\textbf{break}}

\tikzstyle{grammar}=[shorten >=1pt,->,draw=black!50]
\tikzstyle{graph}=[shorten >=1pt,->,draw=black!50]
\tikzstyle{rid} = [align=left, anchor=east]
\tikzstyle{nont}=[rectangle,inner sep=5pt, draw, fill=white, minimum width=20pt]
\tikzstyle{t}=[circle,inner sep=3pt, draw, fill=white, minimum width=15pt]
\tikzstyle{g}=[inner sep=3pt, fill=white]
\tikzstyle{lhs}=[inner sep=5pt, fill=white]
\tikzstyle{w}=[circle, inner sep=2pt, below=8pt, draw, fill=white]
\tikzstyle{edge}=[->, very thick, -latex]
\tikzstyle{edgeLabel}=[midway, above]
\tikzstyle{wedge}=[-, thick]
\tikzstyle{biedge}=[-, very thick]
\tikzstyle{pipe}=[-, very thick]
\tikzstyle{morph}=[-, very thick, dashed, -latex]


\subfigtopskip=0pt
\subfigcapskip=0pt
\subfigbottomskip=0pt

\begin{document}

\title{Model Transformation with Context-free Triple Graph Grammars and Application Conditions}

%\author{\authorname{William da Silva\sup{1,2}, Max Bureck\sup{1} and Ina Schieferdecker\sup{1,2} and Christian Hein\sup{1}}
%\affiliation{\sup{1}Fraunhofer Fokus Institute, Berlin, Germany}
%\affiliation{\sup{2}Technische Universität Berlin, Berlin, Germany}
%\email{\{william.bombardelli.da.silva, max.bureck\}@fokus.fraunhofer.de}
%}

\keywords{Triple Graph Grammars, NCE Graph Grammars, Model Transformation, Model-driven Development.}

%TODO: Between 70 and 200 words
\abstract{}

\onecolumn \maketitle \normalsize \vfill

\section{\uppercase{Introduction}}

\noindent 



\section{\uppercase{Related Works}}

\noindent In this section, we offer a short literary review on the graph grammar and triple graph grammar approaches that are more relevant to our work. We focus, therefore, on the context-free node label replacement approach for graph grammars, although, there is a myriad of different alternatives to it, for example, the algebraic approach \cite{ehrig1999handbook}. We refer to context-free grammars, inspired by the use of such classification for string grammars, in a relaxed way without any compromise to any definition.

%We divide the node label replacement approaches into context-sensitive and context-free approaches, we refer to context-sensitive and context-free grammars, inspired by the use of such classification for string grammars, in a relaxed way without any compromise to any definition of context-freeness for graph grammars. The context-sensitive field includes the \textit{layered graph grammar}, whose semantics consists of the replacement of graphs by other graphs governed by morphisms \cite{rekers1997defining} and for which exponential-time bottom-up parsing algorithms have been proposed \cite{rekers1995graph,bottoni2000efficient,furst2011improving}. Another context-sensitive formalism is the \textit{reserved graph grammar}, that is based on the replacement of directed graphs by necessarily greater directed graphs governed by simple embedding rules \cite{zhang2001context} and for which exponential and polynomial-time bottom-up algorithms have been proposed in \cite{zeng2005rgg+,zou2017partial}.

In the node label replacement context-free formalisms stand out the \textit{node label controlled graph grammar} (NLC) and its successor \textit{graph grammar with neighborhood-controlled embedding} (NCE). NLC is based on the replacement of one vertex by a graph, governed by embedding rules written in terms of the vertex's label \cite{rozenberg1986boundary}. For various classes of these grammars, there exist polynomial-time top-down and bottom-up parsing algorithms \cite{flasinski1993parsing,flasinski2014characteristics,rozenberg1986boundary,wanke1991algorithms}. The recognition complexity and generation power of such grammars have also been analyzed \cite{flasinski1998power,kim2012structure}. NCE occurs in several formulations, including a context-sensitive one, but here we focus on the context-free formulation, where one vertex is replaced by a graph, and the embedding rules are written in terms of the vertex's neighbors \cite{janssens1982graph,skodinis1998neighborhood}. For some classes of these grammars, polynomial-time bottom-up parsing algorithms and automaton formalisms were proposed and analyzed \cite{kim2001efficient,brandenburg2005finite}. In special, one of these classes is the \textit{boundary graph grammar with neighborhood-controlled embedding} (BNCE), that is used in our approach for model transformation.

Regarding TGG \cite{schurr1994specification}, a 20 years review of the realm is put forward by Anjorin et al. \cite{anjorin201620}. In special, advances are made in the direction of expressiveness with the introduction of application conditions \cite{klar2010extended} and of modularization \cite{anjorin2014modularizing}. Furthermore, in the algebraic approach for graph grammars, we have found proposals that introduce inheritance \cite{bardohl2004integrating,hermann2008typed} and variables \cite{hoffmann2005graph} to the formalisms. Nevertheless, we do not know any approach that introduces non-terminal symbols to TGG with the purpose of gaining expressiveness or usability. In this sense, our proposal brings something new to the current state-of-the-art.

\section{\uppercase{Context-free TGG}}

\noindent

\subsection{NCE Graph Grammars}

\noindent

\subsection{NCE Triple Graph Grammars}

\noindent

\section{\uppercase{Context-free TGG with Application Conditions}}

\noindent

\section{\uppercase{Model Transformation with NCE PAC TGG}}

\noindent

\section{\uppercase{Evaluation}}

\noindent

\section{\uppercase{Conclusion}}

\noindent


\vfill
\bibliographystyle{apalike}
{\small
\bibliography{bibliography}}

\vfill
\end{document}

