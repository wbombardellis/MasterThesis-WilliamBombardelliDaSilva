@article{Rozenberg1986,
	abstract = {Node label controlled (NLC) grammars are graph grammars (operating on node labeled undirected graphs) which rewrite single nodes only and establish connections between the embedded graph and the neighbors of the rewritten node on the basis of the labels of the involved nodes only. They define (possibly infinite) languages of undirected node labeled graphs (or, if we just omit the labels, languages of unlabeled graphs). Here we consider a restriction of NLC grammars, so-called boundary NLC (BNLC) grammars, distinguished by the property that whenever in a graph already generated two nodes may be rewritten, then these nodes are not adjacent. The graph languages generated by this type of grammars are called BNLC languages. Although we show that this restriction leads to a smaller class of languages, still enough generative power remains to define interesting graph languages. For example, trees, complete bipartite graphs, maximal outerplanar graphs, k-trees, graphs of bandwidth ≤k, graphs of cyclic bandwidth ≤k, graphs of binary tree bandwidth ≤k, graphs of cutwidth ≤k (always for a fixed positive integer k) turn out all to be BNLC languages. We prove a number of normal forms for BNLC grammars and then we indicate their usefulness by various applications. In particular, we show that for connected graphs of bounded degree the membership problem for BNLC languages is solvable in deterministic polynomial time. {\textcopyright} 1986 Academic Press, Inc.},
	author = {Rozenberg, Grzegorz and Welzl, Emo},
	doi = {10.1016/S0019-9958(86)80045-6},
	file = {:media/dados/2018-1/MasterThesis/Papers/27{\_}BoundaryNLCGG.pdf:pdf},
	issn = {00199958},
	journal = {Information and Control},
	number = {1-3},
	pages = {136--167},
	title = {{Boundary NLC graph grammars-Basic definitions, normal forms, and complexity}},
	volume = {69},
	year = {1986}
}