\documentclass[]{article}

%\usepackage{natbib}

%opening
\title{Properties of Triple Graph Grammars with Non-Terminal Nodes}
\author{}

% Margins
\topmargin=-0.45in
\evensidemargin=0in
\oddsidemargin=0in
\textwidth=6.5in
\textheight=9.0in
\headsep=0.25in 

\begin{document}

\maketitle


\section*{Table of Contents}

\subsection*{Introduction}
Overview of the MDE. Potential and problems of it. The challenge of model transformation. A solution: Triple Graph Grammars (TGG) (justification). A problem of TGG (usability/ amount of grammar rules). Our solution for this problem: TGG with non-terminal nodes. Overview of our approach (graph grammars with non-terminal nodes, NCE grammars, graph language, parsing, transformation). Short summary of the results. Remainder.

\subsection*{Related Work}
TGG main references and reviews. Usability enhancements proposed for TGG. Graph grammar main references. NCE and B-eNCE graph grammar. Alternative proposal of graph grammars.

\subsection*{Theoretical Review}
Definitions: Graph, Grammar (Syntax and semantics), NCE GG, B-eNCE GG, Derivation, Language, TGG (Syntax and semantics). Examples

\subsection*{Parsing of Graphs with B-eNCE Graph Grammars}
Adapted parsing procedure. Examples

\subsection*{Transformation of Graphs with B-eNCE Triple Graph Grammars}
Introduction to the B-eNCE TGG. Transformation procedure. Examples

\subsection*{An extension of B-eNCE Triple Graph Grammars with Look-ahead}
Problem with B-eNCE TGG. Solution with look-ahead. Critical view. Examples

\subsection*{Implementation}
Concrete implementation. Critical view.

\subsection*{Evaluation}
Evaluation of the results (mainly the usability gain of the new formalism B-eNCE TGG in comparison to the state-of-the-art). Critical discussion. Positive and negative aspects of the proposed approach. Future work.

\subsection*{Conclusion}
Summary and closing words.

%\bibliographystyle{authordate1}
%\bibliography{bibliography}

\end{document}
