\noindent
	\begin{tikzpicture}[grammar]
	\node[rid] at (\ridX,\ridY) {$r_0:$};
	\draw (\lhsX,0) node[lhs] (lhs) {K ::=};
	
	\draw (0.5,0) node[t, label=90:$c_0$] (v1) {c};
	\draw (1.2,0) node[t] (v2) {a};
	\draw (2,0) node[nont] (v3) {K};
	\draw (2,-0.5) node[nont] (v4) {K};
	\draw[edge] (v2) -- (v1) node [edgeLabel] {$s$};
	\draw[edge] (v2) -- (v3) node [edgeLabel] {$t$};
	
	%The next rule separator
	\draw[pipe] (2.5,-0.5) -- (2.5,\ridY);
	
	\node[rid] at (3.2,\ridY) {$r_1:$};
	\draw (3.0,0) node[t, label=90:$c_1$] (v5) {c};
	\draw (3.0,-0.5) node[nont] (v6) {K};
	
	%The next rule separator
	\draw[pipe] (3.5,-0.5) -- (3.5,\ridY);
	
	\node[rid] at (4.2,\ridY) {$r_2:$};
	\draw (4,0) node[empty] (v7) {$\emptyTG$};
	\end{tikzpicture}