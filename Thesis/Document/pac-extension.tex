%%%%%%%%%%%%%%%%%%%%%%%%%%%%%%%%%%%%%%%%%%%%%%%%%%%%%%%%%%%%%%%%%%%%%%%%
%%%%%%%%%%%%%%%%%%%%%%%%%%%% PAC EXTENTION %%%%%%%%%%%%%%%%%%%%%%%%%%%%%
%%%%%%%%%%%%%%%%%%%%%%%%%%%%%%%%%%%%%%%%%%%%%%%%%%%%%%%%%%%%%%%%%%%%%%%%

The NCE graph grammar formalism from \cite{janssens1982graph}, presented in the previous sections, can define with very few rules the languages of several classes of labeled graphs, including trees, path graphs, star graphs, control-flow graphs, edgeless graphs, complete graphs, and others. However, it is at least difficult to define the languages of other classes, like the class-diagram graphs, with NCE graph grammars. In this Section, we approach the problem of defining a NCE graph grammar for these classes of graphs and propose a solution for that by means of an extension of NCE that includes application conditions.

Class diagrams are commonly used to model object-oriented software artifact that are composed of several classes related by associations. For the sake of demonstrating the problem of NCE with class diagrams, consider a simplified view of the class-diagrams graphs, in which a vertex has either label $c$ or $a$, respectively representing a class or an association, and an edge between an association and a class with label $s$ ($t$) signalizes that the class is the source (target) of the association. In Figure \ref{fig:classdiagram-g}, a class-diagram graph with two classes connected by two associations is depicted. An attempt for a NCE graph grammar that would describe the language of all class-diagram graphs is $GG = (\{K,a,c,s,t\}, \{a,c,s,t\}, K, \{r_0, r_1, r_2\})$, with $r_0$, $r_1$, and $r_2$ depicted in Figure \ref{fig:classdiagram-gg} and $\omega_0(c_0) = \omega_1(c_1) = \{(t,a)\}$ and $\omega_0 = \emptyset$ being the complete embedding definition of the rules $r_0$, $r_1$, and $r_2$, respectively.

\begin{figure}[h]
	\begin{subfigure}[t]{0.5\textwidth}
		\centering
		\begin{tikzpicture}[graph]
	\draw (0,0) node[t] (v1) {c};
	\draw (1,0) node[t] (v2) {a};
	\draw (1,-0.5) node[t] (v4) {a};
	\draw (2,0) node[t] (v3) {c};
	\draw[edge] (v2) -- (v1) node [edgeLabel] {$s$};
	\draw[edge] (v2) -- (v3) node [edgeLabel] {$t$};
	\draw[edge] (v4) -- (v3) node [midway, below] {$s$};
	\draw[edge] (v4) -- (v1) node [midway, below] {$t$};
\end{tikzpicture}
		\caption{A class-diagram graph}
		\label{fig:classdiagram-g}
	\end{subfigure}
	\begin{subfigure}[t]{0.49\textwidth}
		\centering
		\noindent
	\begin{tikzpicture}[grammar]
	\node[rid] at (\ridX,\ridY) {$r_0:$};
	\draw (\lhsX,0) node[lhs] (lhs) {K ::=};
	
	\draw (0.5,0) node[t, label=90:$c_0$] (v1) {c};
	\draw (1.2,0) node[t] (v2) {a};
	\draw (2,0) node[nont] (v3) {K};
	\draw (2,-0.5) node[nont] (v4) {K};
	\draw[edge] (v2) -- (v1) node [edgeLabel] {$s$};
	\draw[edge] (v2) -- (v3) node [edgeLabel] {$t$};
	
	%The next rule separator
	\draw[pipe] (2.5,-0.5) -- (2.5,\ridY);
	
	\node[rid] at (3.2,\ridY) {$r_1:$};
	\draw (3.0,0) node[t, label=90:$c_1$] (v5) {c};
	\draw (3.0,-0.5) node[nont] (v6) {K};
	
	%The next rule separator
	\draw[pipe] (3.5,-0.5) -- (3.5,\ridY);
	
	\node[rid] at (4.2,\ridY) {$r_2:$};
	\draw (4,0) node[empty] (v7) {$\emptyTG$};
	\end{tikzpicture}
		\caption{The depiction of the rules of $GG$}
		\label{fig:classdiagram-gg}
	\end{subfigure}
	\caption{An example for a class-diagram graph with two classes connected by associations in (a) and the rules $r_0$, $r_1$, and $r_2$ of the graph grammar $GG$ in (b)}
\end{figure}

The problem of the graph grammar $GG$ is that it does not define the complete language of the class-diagram graph. In fact, the graph in Figure \ref{fig:classdiagram-g} is not in $L(GG)$. To see this, consider the following derivation in $GG$

\begin{equation*}
	\startG{GG} \deriv{r_0}{v_0}{} 
	\begin{tikzpicture}[graph]
	\draw (0.5,0) node[t, label=90:$v_1$] (v1) {c};
	\draw (1.2,0) node[t] (v2) {a};
	\draw (2,0) node[nont, label=90:$K_0$] (v3) {K};
	\draw (2,-0.5) node[nont, label=270:$K_1$] (v4) {K};
	\draw[edge] (v2) -- (v1) node [edgeLabel] {$s$};
	\draw[edge] (v2) -- (v3) node [edgeLabel] {$t$};
	\end{tikzpicture}
	\deriv{r_2}{K_1}{} 
	\begin{tikzpicture}[graph]
	\draw (0.5,0) node[t, label=90:$v_1$] (v1) {c};
	\draw (1.2,0) node[t] (v2) {a};
	\draw (2,0) node[nont, label=90:$K_0$] (v3) {K};
	\draw[edge] (v2) -- (v1) node [edgeLabel] {$s$};
	\draw[edge] (v2) -- (v3) node [edgeLabel] {$t$};
	\end{tikzpicture}
	\deriv{r_0}{K_0}{}
	\begin{tikzpicture}[graph]
	\draw (0.5,0) node[t, label=90:$v_1$] (v1) {c};
	\draw (1.2,0) node[t] (v2) {a};
	\draw (2,0) node[t] (v3) {c};
	\draw (1.2,-0.5) node[t, label=270:$v_2$] (v4) {a};
	\draw (0.5,-0.5) node[nont, label=270:$K_2$] (v5) {K};
	\draw (2,-0.5) node[nont, label=270:$K_3$] (v6) {K};
	\draw[edge] (v2) -- (v1) node [edgeLabel] {$s$};
	\draw[edge] (v2) -- (v3) node [edgeLabel] {$t$};
	\draw[edge] (v4) -- (v3) node [midway, below] {$s$};
	\draw[edge] (v4) -- (v5) node [edgeLabel] {$t$};
	\end{tikzpicture}
	\deriv{r_2}{K_3}{}
	\begin{tikzpicture}[graph]
	\draw (0.5,0) node[t, label=90:$v_1$] (v1) {c};
	\draw (1.2,0) node[t] (v2) {a};
	\draw (2,0) node[t] (v3) {c};
	\draw (1.2,-0.5) node[t, label=270:$v_2$] (v4) {a};
	\draw (0.5,-0.5) node[nont, label=270:$K_2$] (v5) {K};
	\draw[edge] (v2) -- (v1) node [edgeLabel] {$s$};
	\draw[edge] (v2) -- (v3) node [edgeLabel] {$t$};
	\draw[edge] (v4) -- (v3) node [midway, below] {$s$};
	\draw[edge] (v4) -- (v5) node [edgeLabel] {$t$};
	\end{tikzpicture}
\end{equation*}

This is the closest we get to deriving the graph in Figure \ref{fig:classdiagram-g} using GG. Thereby, we would like to connect the association $v_2$ to the class $v_1$ but it is not possible, because $v_1$ was not a neighbor of the vertex $K_0$ that preceded $v_2$. Notice that a vertex in any sentential form can only be either connected to vertices that stem from the same rule application or to neighbors of its precedent vertex. In fact, this seems to be a general characteristic for context-free grammars, where the information about elements in the context of the precedents are not available for descendant elements. In order to overcome it, one could potentially elaborate an alternative grammar that defines the desired language completely and concisely, but we believe that such ad-hoc solution would include a bigger number of rules and add complexity to the grammar. With that in mind, we propose in the sequel an extension of the NCE grammar formalism with positive application conditions (PAC) that solves this issue. 

In NCE graph grammars with PAC, rules' right-hand sides are equipped with application conditions in form of special vertices that are produced by derivation steps and removed by so-called resolution steps. A resolution step is responsible for removing such special vertices and moving their adjacent edges to other vertices. This resolution mechanism allows that the vertex $v_2$ from the previous example be connected to $v_1$. 

In order to define the PAC mechanism in detail, the definitions of rule and derivation step are augmented as follows.

\begin{definition}
	A \emph{rule with PAC} is of the form $(A \pro R, \omega, U)$ with $A$, $R$ and $\omega$ as described in \ref{def:gg} and $U \subseteq \{v \in V_R \st \phi_R(v) \in \Delta\}$, the set of special vertices, called PAC vertices.
\end{definition}

If a graph grammar has at least one rule with PAC, then we say it is a graph grammar with PAC.

\begin{definition}
	A \textit{concrete derivation step with PAC} in the graph grammar $GG$ is of the form $G \cderivpac{r}{v}{U}{GG} H$ with $G$, $H$, $v$ being as described in Definition \ref{def:gg_dstep}, and $r = (A \pro R, \omega, U)$ being a production rule with PAC. Given that, a \textit{derivation step with PAC} is, analogously, of the form $G \derivpac{r}{v}{W}{GG} H'$ with $W = m(U)$ where $m$ is the isomorphism from $H$ and $H'$.
\end{definition}

So far, PAC vertices do not change anything in the behavior of a derivation step and the set $U$ in a derivation step serves just to tag which vertices are PAC in a sentential form. Nevertheless, PAC vertices play an important role on a resolution step, defined below. If $W$ is empty, we might omit it from the notation.

\begin{definition}
	\label{def:resolv}
	Let $GG = (\Sigma, \Delta, S, P)$ be a graph grammar and $G$ a graph over $\Delta$, $G$ \emph{resolves into} $H$ with the resolution partial function $\rho: V_G \pto V_G$, we write $G \resolv{\rho} H$ and call it a \emph{resolution step}, if, and only if, the following holds:
	\begin{align*}
		& \forall v \in \dom \rho \. \rho(v) \notin \dom \rho \text{ and } \phi_G(\rho(v)) = \phi_G(v) \text{ and}\\
		V_H & = V_G \setminus \dom \rho \text{ and}\\
		E_H & = (E_G \setminus (\{(u,l,t) \st u \in \dom \rho, (u,l,t) \in E_G\} \\
			&\hspace{3.5em} \cup \{(t,l,u) \st u \in \dom \rho, (t,l,u) \in E_G\}))\\
		    & \cup \{(\rho(u),l,t) \st u \in \dom \rho, (u,l,t) \in E_G\} \\
		    & \cup \{(t,l,\rho(u)) \st u \in \dom \rho, (t,l,u) \in E_G\}
	\end{align*}
\end{definition}

A resolution step can be informally understood as the removal of the PAC vertices of $G$--- that are in the domain of the resolution function $\rho$ ---followed by the redirection of the edges adjacent to the PAC vertices to other vertices of $H$.

For the PAC mechanism to work, it is still necessary to combine derivation and resolution steps to define the language of a grammar with PAC, what we do in the following.

\begin{definition}
	A \emph{production} $Q$ in a graph grammar with PAC is a sequence of $n$ derivation steps followed by $n$ resolution steps with $n > 0$, as follows:
	\begin{equation*}
		Q = (G_0 \derivpac{r_0}{v_0}{W_0}{} G_1 \derivpac{r_1}{v_1}{W_1}{} \dots \derivpac{r_{n-1}}{v_{n-1}}{W_{n-1}}{} G_n^0 \resolv{\rho_0} G_n^1 \resolv{\rho_1} \dots \resolv{\rho_{n-1}} G_n^n)
	\end{equation*}
	with $\rho_i$ being a resolution total function $\rho_i : m_i(W_i) \to V_{G_n^i}$ and $m_i : W_i \to V_{G_n^i}$ the mapping from the PAC vertices generated on the derivation step $i$ to their correspondent vertices in $G_n^i$, for all $0 \le i < n$.
\end{definition}

It is clear that, the mapping $m$ of the previous definition exists and is bijective because all PAC vertices are, by definition, terminal and, therefore, are not deleted by derivation steps and, moreover, the images of all $m_i$ are pair-wise disjunct.

\begin{definition}
	The \emph{language} $L(GG)$ generated by the grammar $GG$ with PAC is
	\begin{equation*}
		L(GG) = \{ H \text{ is a graph over } \Delta \text{ and } \startG{GG} \derivpacn{}{n} H' \resolvn{}{n} H \}
	\end{equation*}
	where $\derivpacn{}{n}$ and $\resolvn{}{n}$ denote a sequence of $n$ derivation steps and $n$ resolution steps, respectively.
\end{definition}

Ultimately, we put forward a NCE graph grammar with PAC whose language is the set of all class-diagram graphs. This grammar is $GG = (\{K, A, a, c, s, t\}, \{a, c, s, t\}, K, \{r_0, r_1, r_2, r_3\})$ with $\omega_0(c_0) = \{(t,a)\}$, $\omega_2(a_2) = \{(s,c)\}$, $\omega_2(c_2) = \{(s,a),(t,a)\}$, $\omega_1 = \omega_3 = \emptyset$ being the complete characterization of the embedding functions of the respective rules, and the the rules being denoted as below. We advise that PAC vertices and their adjacent edges are depicted with dotted lines.

\noindent
\begin{tikzpicture}[grammar]
	\node[rid] at (\ridX,\ridY) {$r_0:$};
	\draw (\lhsX,0) node[lhs] (lhs) {K ::=};
	
	\draw (0.2,0) node[t, label=90:$c_0$] (v1) {c};
	\draw (1,0) node[nont] (v2) {A};
	\draw (1.7,0) node[nont] (v3) {K};
	\draw[edge] (v2) -- (v1) node [edgeLabel] {$s$};
	
	%The next rule separator
	\draw[pipe] (2.2,-0.5) -- (2.2,\ridY);
	
	\node[rid] at (2.8,\ridY) {$r_1:$};
	\draw (2.5,0) node[empty] (v7) {$\emptyTG$};
\end{tikzpicture}
\begin{tikzpicture}[grammar]
\node[rid] at (\ridX,\ridY) {$r_2:$};
\draw (\lhsX,0) node[lhs] (lhs) {A ::=};

\draw (0.2,0) node[t, label=90:$a_2$] (v1) {a};
\draw (1,0) node[pac, label=90:$c_2$] (v2) {c};
\draw[pacedge] (v1) -- (v2) node [edgeLabel] {$t$};

%The next rule separator
\draw[pipe] (1.6,-0.5) -- (1.6,\ridY);

\node[rid] at (2.2,\ridY) {$r_3:$};
\draw (2,0) node[empty] (v7) {$\emptyTG$};
\end{tikzpicture}

Below, we demonstrate that the graph from Figure \ref{fig:classdiagram-g} is in $L(GG)$, by means of a production in $GG$.

\begin{align*}
	\startG{GG} \derivpac{r_0}{v_0}{\emptyset}{} &
	\begin{tikzpicture}[graph]
	\draw (0.5,0) node[t, label=90:$v_1$] (v1) {c};
	\draw (1.2,0) node[nont, label=90:$A_0$] (v2) {A};
	\draw (2,0) node[nont, label=90:$K_0$] (v3) {K};
	\draw[edge] (v2) -- (v1) node [edgeLabel] {$s$};
	\end{tikzpicture}
	\derivpac{r_0}{K_0}{\emptyset}{} &
	\begin{tikzpicture}[graph]
	\draw (0.5,0) node[t, label=90:$v_1$] (v1) {c};
	\draw (1.2,0) node[nont, label=90:$A_0$] (v2) {A};
	\draw (2,0) node[t, label=90:$v_2$] (v3) {c};
	\draw (1.2,-0.5) node[nont, label=270:$A_1$] (v4) {A};
	\draw (0.5,-0.5) node[nont, label=270:$K_1$] (v5) {K};
	\draw[edge] (v2) -- (v1) node [edgeLabel] {$s$};
	\draw[edge] (v4) -- (v3) node [midway, below] {$s$};
	\end{tikzpicture} \\
	%
	\derivpac{r_1}{K_1}{\emptyset}{} &
	\begin{tikzpicture}[graph]
	\draw (0.5,0) node[t, label=90:$v_1$] (v1) {c};
	\draw (1.2,0) node[nont, label=90:$A_0$] (v2) {A};
	\draw (2,0) node[t, label=90:$v_2$] (v3) {c};
	\draw (1.2,-0.5) node[nont, label=270:$A_1$] (v4) {A};
	\draw[edge] (v2) -- (v1) node [edgeLabel] {$s$};
	\draw[edge] (v4) -- (v3) node [midway, below] {$s$};
	\end{tikzpicture}
	\derivpac{r_2}{A_0}{\{p_0\}}{} &
	\begin{tikzpicture}[graph]
	\draw (0.5,0) node[t, label=90:$v_1$] (v1) {c};
	\draw (1.2,0) node[t] (v2) {a};
	\draw (1.2,0.7) node[pac, label=0:$p_0$] (v5) {c};
	\draw (2,0) node[t, label=90:$v_2$] (v3) {c};
	\draw (1.2,-0.5) node[nont, label=270:$A_1$] (v4) {A};
	\draw[edge] (v2) -- (v1) node [edgeLabel] {$s$};
	\draw[pacedge] (v2) -- (v5) node [vredgeLabel] {$t$};
	\draw[edge] (v4) -- (v3) node [midway, below] {$s$};
	\end{tikzpicture}
\end{align*}
\begin{align*}
	\derivpac{r_2}{A_1}{\{p_1\}}{} &
	\begin{tikzpicture}[graph]
	\draw (0.5,0) node[t, label=90:$v_1$] (v1) {c};
	\draw (1.2,0) node[t] (v2) {a};
	\draw (1.2,0.7) node[pac, label=0:$p_0$] (v5) {c};
	\draw (2,0) node[t, label=90:$v_2$] (v3) {c};
	\draw (1.2,-0.5) node[t] (v4) {a};
	\draw (0.5,-0.5) node[pac, label=270:$p_1$] (v6) {c};
	\draw[edge] (v2) -- (v1) node [edgeLabel] {$s$};
	\draw[pacedge] (v2) -- (v5) node [vredgeLabel] {$t$};
	\draw[edge] (v4) -- (v3) node [midway, below] {$s$};
	\draw[pacedge] (v4) -- (v6) node [edgeLabel] {$t$};
	\end{tikzpicture}
	\;\;\; \resolvn{\emptyset}{}{3} \;\;\; &
	\begin{tikzpicture}[graph]
	\draw (0.5,0) node[t, label=90:$v_1$] (v1) {c};
	\draw (1.2,0) node[t] (v2) {a};
	\draw (1.2,0.7) node[pac, label=0:$p_0$] (v5) {c};
	\draw (2,0) node[t, label=90:$v_2$] (v3) {c};
	\draw (1.2,-0.5) node[t] (v4) {a};
	\draw (0.5,-0.5) node[pac, label=270:$p_1$] (v6) {c};
	\draw[edge] (v2) -- (v1) node [edgeLabel] {$s$};
	\draw[pacedge] (v2) -- (v5) node [vredgeLabel] {$t$};
	\draw[edge] (v4) -- (v3) node [midway, below] {$s$};
	\draw[pacedge] (v4) -- (v6) node [edgeLabel] {$t$};
	\end{tikzpicture} \\
	%
	\;\;\; \resolv{\{p_0 \mapsto v_2\}}{} &
	\begin{tikzpicture}[graph]
	\draw (0.5,0) node[t, label=90:$v_1$] (v1) {c};
	\draw (1.2,0) node[t] (v2) {a};
	\draw (2,0) node[t, label=90:$v_2$] (v3) {c};
	\draw (1.2,-0.5) node[t] (v4) {a};
	\draw (0.5,-0.5) node[pac, label=270:$p_1$] (v6) {c};
	\draw[edge] (v2) -- (v1) node [edgeLabel] {$s$};
	\draw[edge] (v2) -- (v3) node [edgeLabel] {$t$};
	\draw[edge] (v4) -- (v3) node [midway, below] {$s$};
	\draw[pacedge] (v4) -- (v6) node [edgeLabel] {$t$};
	\end{tikzpicture}
	\resolv{\{p_1 \mapsto v_1\}}{} \;\;\; &
	\begin{tikzpicture}[graph]
	\draw (0.5,0) node[t, label=90:$v_1$] (v1) {c};
	\draw (1.2,0) node[t] (v2) {a};
	\draw (2,0) node[t, label=90:$v_2$] (v3) {c};
	\draw (1.2,-0.5) node[t] (v4) {a};
	\draw[edge] (v2) -- (v1) node [edgeLabel] {$s$};
	\draw[edge] (v2) -- (v3) node [edgeLabel] {$t$};
	\draw[edge] (v4) -- (v3) node [midway, below] {$s$};
	\draw[edge] (v4) -- (v1) node [midway, below] {$t$};
	\end{tikzpicture}
\end{align*}

In this production, the rule $r_2$ creates, through two applications, two PAC vertices $p_0$ and $p_1$, which are then removed and have their adjacent vertices moved to the vertices $v_2$ and $v_1$, respectively, through the last two resolution steps. The resolution steps have thus the power of connecting vertices that could not be connected otherwise.

Although it is difficult to interpret PAC vertices as application conditions under this semantics of production and resolution steps, we still maintain this nomenclature, for we think that our PAC mechanism ensembles quite nearly positive application conditions of standard TGG, except that it does not require the condition to hold at the moment of the rule application, instead it can hold later on in the derivation, whereas the PAC of standard TGG requires the PAC vertex to be matched at the moment of a rule application. In other words, a PAC vertex in a rule of a NCE TGG can be understood as a condition that requires the creation of such a vertex before or after the application of the rule.

To finish, consider these two following remarks.

\begin{remark}
	If, for all rules $(A \pro R,\omega,U)$ in a grammar $GG$ have $U = \emptyset$, then the $GG$ degrades to a normal NCE grammar without PAC and the resolution steps have no effect in $L(GG)$.
\end{remark}

\begin{remark}
	Given a graph grammar $GG$ with PAC, if the graph $g$ is in $L(GG)$, then $g$ has no PAC vertices. That is, the resolution steps remove all PAC vertices, because every resolution function $\rho_i$ is required to map to vertices that are not in its domain. This guarantees that the number of PAC vertices reduces at each resolution step with $\rho_i \neq \emptyset$.
\end{remark}

%%%==================================================================%%%
%%%%%%%%%%%%%%%%%%%%%%%%%%% PARSING %%%%%%%%%%%%%%%%%%%%%%%%%%%%%%%%%%%%
%%%==================================================================%%%
\section{Parsing of Graphs with Application Conditions}
Regarding the parsing procedure, the Algorithm \ref{alg:parse} can be slightly modified to support PAC, by augmenting the zone vertices with PAC vertices, changing the way how zone graphs are constructed and how zone vertices are added to $bup$ and to the parsing forest. The details of these changes are described in the sequel.

\begin{definition}
	A \emph{zone vertex with PAC} of a graph $G$ is a triple $(\sigma, U, W)$, with $\sigma$ and $U$ being as explained in Definition \ref{def:zv} and $W \in V_G$ being the set of PAC vertices disjunct from $U$.
\end{definition}

\begin{definition}
	Let $H = \{(\sigma_o,U_0,W_0), \dots, (\sigma_m, U_m, W_m)\})$ be a set of zone vertices with PAC of a graph $G$, as given in Definition \ref{def:z}, and $W(H) = \bigcup_{0 \le i \le m}{W_i}$. A \emph{zone graph with PAC} $Z(H)$ for $H$ is $(V,E,\psi)$, with
	\begin{align*}
		V & = H \cup \{(\psi_G(x),\{x\}, \emptyset) \st x \in \neigh{G}(V(H)) \setminus W(H) \} \\
		E & = \{((\sigma,U,W),l,(\eta,T,X)) \st (\sigma,U,W),(\eta,T,X) \in V \text{ and } U \neq T \text{ and } \\
		& (u,l,t) \in E_G \text{ and } u \in U \setminus X \text{ and } t \in T \setminus W \} \\
		\phi & = \{(\sigma,U,W) \mapsto \sigma \st (\sigma,U,W) \in V\}
	\end{align*}
\end{definition}

%TODO: Probably need to explain better Q, based on the explanation about D
The Algorithm \ref{alg:parsepac} is a parsing method that returns a valid derivation if, and only if, the input graph $G$ is in the language of the graph grammar $GG$ with PAC. Notice that, this procedure does not return a derivation, but a production, that is built by the function $Q$ that performs, analogously to $D$ in Algorithm \ref{alg:parse}, a depth-first walk in the parsing tree.

\begin{algorithm}[!h]
	\caption{Parsing Algorithm for NP BNCE Graph Grammars with PAC}
	\begin{algorithmic}[1]
		\Require $GG \text{ is a valid NP BNCE graph grammar with PAC}$
		\Require $G \text{ is a valid graph over } \Delta$
		\Function{$parse$}{$GG=(\Sigma, \Delta, S, P), G=(V_G,E_G,\phi_G)$}{$:Derivation$}
		\State $bup \gets \{(\phi_G(x),\{x\},\emptyset) \st x \in V_G\}$
		\State $pf \gets \{ \ptree{b}{}{\emptyset} \st b \in bup \}$
		\Repeat
		\State $h \gets \text{\Select } \{X \subseteq bup \st\text{for all } U_i, U_j \in X \text{ with } i \neq j \. U_i \cap U_j = \emptyset \}$
		\ForAll{$d \in \Gamma$}
		\State $r \gets \text{any } \{(d \pro R,\omega,U) \in P \st R \isomorph Y(h) \}$
		\State $l \gets (d,V(h)\setminus W(h), W(h))$ \Comment{$l$ is augmented with PAC $W(h)$}
		\If{$Z(\{l\}) \derivpac{r}{l}{W}{} Z(h)$} \Comment{derivation with PAC is possible}
		\State $bup \gets bup \cup \{l\}$
		\State $pf \gets pf \cup \{ \ptree{l}{r}{\{\ptree{z}{y}{X} \st \ptree{z}{y}{X} \in pf, z \in h \}} \}$
		\EndIf
		\EndFor
		\Until{$(S, V_G, \_) \in bup$} \Comment{if found the root, no matter which PAC}
		\State \Return $(S, V_G, \_) \in bup \? \Just Q(\ptree{(S,V_G, \_)}{y}{X} \in pf) \: \Nothing $
		\EndFunction
		\Ensure $return \text{ is either } \Nothing \text{ or of the form } \Just \startG{GG} \derivpactr{} G \resolvtr{}$
	\end{algorithmic}
	\label{alg:parsepac}
\end{algorithm}

The most important difference between Algorithm \ref{alg:parse} and \ref{alg:parsepac} are, first, in the use of a derivation with PAC $Z(\{l\}) \derivpac{r}{l}{W}{} Z(h)$ in line 9, where $W$ is the set of PAC vertices $U$ from the rule $r$ mapped to the zone graph $Z(h)$, and, second, in the construction of the zone vertex $l$, that is augmented with the PAC vertices $W(h)$ which are, in turn, removed from the normal vertices of $l$, in line 8. In practice, this allows, on the one hand, that PAC vertices participate in the search for rules that produce the desired zone graphs, and, on the other hand, that they be not included in the set of normal vertices of zone vertices so they can be effectively added to the zone vertices that effectively produce them.

%%%==================================================================%%%
%%%%%%%%%%%%%%%%%%%%%% MODEL TRANSFORMATION %%%%%%%%%%%%%%%%%%%%%%%%%%%%
%%%==================================================================%%%
\section{Model Transformation with Application Condition}
In regard to the application of NCE graph grammars with PAC to the problem of model transformation, the extension is also possible, as shown in the next argumentation. First, consider the extension of NCE TGG to support PAC. 

\begin{definition}
	A \emph{triple rule with PAC} is of the form $(A \pro (R_s \ms{} R_c \mt{} R_t), \omega_s, \omega_t, U_s, U_t)$ with $A \pro (R_s \ms{} R_c \mt{} R_t)$, $\omega_s$, and $\omega_t$ as defined in Definition \ref{def:tgg} and $U_s \subseteq \{v \in V_{R_s} \st \phi_{R_s}(v) \in \Delta\}$ the set of PAC vertices of $R_s$ and $U_t \subseteq \{v \in V_{R_t} \st \phi_{R_t}(v) \in \Delta\}$ the set of PAC vertices of $R_t$.
\end{definition}

Analogously, the concepts of \textit{concrete derivation step} and \textit{derivation step} for TGG are extended to be as follows.

\begin{definition}
 A \textit{concrete derivation step with PAC} in the triple graph grammar $TGG$ is of the form $G \tcderivpac{r}{v_s,v_c,v_t}{U_s,U_t}{TGG} H$, where $G$, $H$, $v_s$, $v_c$, $v_t$ being as described in Definition \ref{def:tgg_dstep}, $r = (A \pro (R_s \ms{} R_c \mt{} R_t), \omega_s, \omega_t, U_s, U_t)$ being a triple rule with PAC. Given that, a \textit{derivation step with PAC} is, analogously, of the form $G \tderivpac{r}{v_s,v_c,v_t}{W_s,W_t}{TGG} H'$ with $W_s = m(U_s)$ and $W_t = m(U_t)$ where $m$ is the triple isomorphism from $H$ to $H'$.
\end{definition}

If $W_s$ and $W_t$ are empty, we might omit them from the notation. The \textit{resolution step} for TGG is as follows.

\begin{definition}
	\label{def:tresolv}
	Let $TGG = (\Sigma, \Delta, S, P)$ be a triple graph grammar and $G = (G_s \ms{gs} G_c \mt{gt} G_t)$ a triple graph over $\Delta$, $G$ \emph{resolves into} $H = (H_s \ms{hs} H_c \mt{ht} H_t)$ with the resolution partial functions $\rho_s: V_{G_s} \pto V_{G_s}$ and $\rho_t: V_{G_t} \pto V_{G_t}$, we write $G \tresolv{\rho_s, \rho_t} H$ and call it a \emph{resolution step}, if, and only if, the following holds:
	\begin{align*}
	& \forall v \in \dom \rho_s \. \rho_s(v) \notin \dom \rho_s \text{ and } \phi_{G_s}(\rho_s(v)) = \phi_{G_s}(v) \text{ and}\\
	& \forall v \in \dom \rho_t \. \rho_t(v) \notin \dom \rho_t \text{ and } \phi_{G_t}(\rho_t(v)) = \phi_{G_t}(v) \text{ and}\\
	V_{H_s} & = V_{G_s} \setminus \dom  \rho_s\text{ and}\\
	V_{H_c} & = V_{G_c} \setminus gs^{-1}(\dom  \rho_s) \text{ and}\\
	V_{H_t} & = V_{G_t} \setminus \dom  \rho_t\text{ and}\\
	E_{H_s} & = (E_{G_s} \setminus (\{(u,l,t) \st u \in \dom \rho_s, (u,l,t) \in E_{G_s}\} \\
			&\hspace{3.5em} \cup \{(t,l,u) \st u \in \dom \rho_s, (t,l,u) \in E_{G_s}\}))\\
	& \cup \{(\rho_s(u),l,t) \st u \in \dom \rho_s, (u,l,t) \in E_{G_s}\} \\
	& \cup \{(t,l,\rho_s(u)) \st u \in \dom \rho_s, (t,l,u) \in E_{G_s}\} \\
	E_{H_c} & = E_{G_c} \setminus (\{(u,l,t) \st u \in gs^{-1}(\dom \rho_s), (u,l,t) \in E_{G_c}\} \\
			&\hspace{3.5em} \cup \{(t,l,u) \st u \in gs^{-1}(\dom \rho_s), (t,l,u) \in E_{G_c}\})\\
	E_{H_t} & = (E_{G_t} \setminus (\{(u,l,t) \st u \in \dom \rho_t, (u,l,t) \in E_{G_t}\} \\
	&\hspace{3.5em} \cup \{(t,l,u) \st u \in \dom \rho_t, (t,l,u) \in E_{G_t}\}))\\
	& \cup \{(\rho_t(u),l,t) \st u \in \dom \rho_t, (u,l,t) \in E_{G_t}\} \\
	& \cup \{(t,l,\rho_t(u)) \st u \in \dom \rho_t, (t,l,u) \in E_{G_t}\}
	\end{align*}
\end{definition}

Finally, production and language for TGG are extended as follows.

\begin{definition}
	A \emph{production} $Q$ in a TGG with PAC is a sequence of $n$ derivation steps followed by $n$ resolution steps with $n > 0$, as follows:
	\begin{equation*}
		Q = (G_0 \derivpac{r_0}{s_0,c_0,t_0}{W_0,T_0}{} G_1 \derivpac{r_1}{s_1,c_1,t_1}{W_1,T_1}{} \dots \derivpac{r_{n-1}}{s_{n-1},c_{n-1},t_{n-1}}{W_{n-1},T_{n-1}}{} G_n^0 \resolv{\rho_0,\tau_0} G_n^1 \resolv{\rho_1,\tau_1} \dots \resolv{\rho_{n-1},\tau_{n-1}} G_n^n)
	\end{equation*}
	with $\rho_i: m_i(W_i) \to V_{G_{s,n}^i}$ and $\tau_i: n_i(T_i) \to V_{G_{t,n}^i}$ being the resolution total functions and $m_i : W_i \to V_{G_{s,n}^i}$ and $n_i : T_i \to V_{G_{t,n}^i}$ the mappings from the PAC vertices generated on the derivation step $i$ to their correspondent vertices in the source and target graphs of the triple graph $G_n^i$, for all $0 \le i < n$.
\end{definition}

\begin{definition}
	The \emph{language} $L(TGG)$ generated by the triple grammar $TGG$ with PAC is
	\begin{equation*}
		L(TGG) = \{ H \text{ is a triple graph over } \Delta \text{ and } \startTG{TGG} \tderivpacn{}{n} H' \tresolvn{}{n} H \}
	\end{equation*}
	where $\tderivpacn{}{n}$ and $\tresolvn{}{n}$ denote a sequence of $n$ derivation steps and $n$ resolution steps, respectively.
\end{definition}

To define the model transformation procedure, consider the redefinition of the $\source$ function.

\begin{definition}
	\label{def:sourcepac}
	Let $r = (A \pro (G_s \ms{} G_c \mt{} G_t), \omega_s, \omega_t, U_s, U_t)$ be a production rule of a triple graph grammar, redefine $\source(r)$ as $\source(r) = (A \pro G_s,\omega_s,U_s)$ and $\source^{-1}((A \pro G_s,\omega_s,U_s)) = r$.
\end{definition}

Furthermore, consider the definition of the PAC consistent (PC) property of TGG, which assures that a PAC vertex of the source graph is connected with a PAC vertex in the correspondence and in the target graphs.

\begin{definition}
	Let $TGG = (\Sigma, \Delta, S, P)$ be a triple graph grammar and $\Pi = \bigcup_{r \in P}{\phi_{G_s}(U_s)}$ be the set of all PAC labels, $TGG$ is \emph{PAC consistent} (PC) if, and only if, for all rules $(A \pro (G_s \ms{ms} G_c \mt{mt} G_t), \omega_s, \omega_t, U_s, U_t) \in P$, the following holds
	\begin{enumerate}
		\item $\forall v \in U_s \. mt(ms^{-1}(v)) \in U_t$
		\item $\forall v \in G_s \. \text{if } \phi_{G_s}(v) \in \Pi \text{ then } mt(ms^{-1}(v)) \in G_t$
	\end{enumerate}
\end{definition}

%TODO: Remember ms mt are bijective, thus PC is well defined

Finally, Theorem \ref{thm:one_d_enough_pac} is, analogously to Theorem \ref{thm:one_d_enough}, allows us to construct a production in $TGG$ out of a production in $S(TGG)$, for a triple graph grammar $TGG$.

\begin{theorem}
	\label{thm:one_d_enough_pac}
	Let $TGG = (\Sigma, \Delta, S, P)$ be a NTC PC TGG and $k \ge 1$,
	\begin{equation*}
		Q = \startTG{TGG} \tderivpac{r_0}{s_0,c_0,t_0}{W_0,Y_0}{} G^1 \tderivpac{r_1}{s_1,c_1,t_1}{W_1,Y_1}{} \dots \tderivpac{r_{k-1}}{s_{k-1},c_{k-1},t_{k-1}}{W_{k-1},Y_{k-1}}{} G^k \tresolv{\rho_0,\tau_0} H^1 \tresolv{\rho_1,\tau_1} \dots \tresolv{\rho_{k-1},\tau_{k-1}} H^k
	\end{equation*}
	is a production in $TGG$ if, and only if,
	\begin{equation*}
		\overline{Q} = \startG{S(TGG)} \derivpac{s(r_0)}{s_0}{W_0}{} G^1_s \derivpac{s(r_1)}{s_1}{W_1}{} \dots \derivpac{s(r_{k-1})}{s_{k-1}}{W_{k-1}}{} G^k_s \tresolv{\rho_0} H_s^1 \tresolv{\rho_1} \dots \tresolv{\rho_{k-1}} H_s^k
	\end{equation*}
	is a production in $S(TGG)$.
\end{theorem}
\begin{proof}
	We want to show that if $Q$ is a production in $TGG = (\Sigma, \Delta, S, P)$, then $\overline{Q}$ is a production in $SG := S(TGG) = (\Sigma, \Delta, S, SP)$, and vice-versa.
	
	For the first half of the production, that is, for the $k$ derivations steps the equivalence holds trivially because the PAC vertices $W_i$ and $Y_i$ do not harm the result of Theorem \ref{thm:one_d_enough}. So we just have to show it for the second half, that is, the $k$ resolutions, what we do by induction in the following.
	
	The induction base is trivial, because the start graph $\startTG{TGG}$ has no PAC vertices, hence, $W_0 = Y_0 = \emptyset$ and thus $\rho_0 = \tau_0 = \emptyset$. Therefore, if $G^k \tresolv{\rho_0,\tau_0} H^1$, then $G^k_s \tresolv{\rho_0} H_s^1$, and vice-versa.
	
	For the induction step, we want to show that if $G^k \tresolvtr{} H^i \tresolv{\rho_{i},\tau_{i}} H^{i+1}$ is a resolution, then $G^k_s \tresolvtr{} H^i_s \tresolv{\rho_{i}} H^{i+1}_s$ is also a resolution and vice-versa, provided that the equivalence holds for the first $i$ steps, so we just have to show it for the step $i+1$.
	
	So, in the one direction, if $H^i \tresolv{\rho_{i},\tau_{i}} H^{i+1}$, then, trivially, $H^i_s \tresolv{\rho_{i}} H^{i+1}_s$, by Definition \ref{def:resolv} and \ref{def:tresolv}.
	
	In the other direction, we want to show that if $H^i_s \tresolv{\rho_{i}} H^{i+1}_s$ then $(H^i_s \ms{ms} H^i_c \mt{mt} H^i_t) \tresolv{\rho_i,\tau_i} (H^{i+1}_s \ms{} H^{i+1}_c \mt{} H^{i+1}_t)$. For that regard, we set 
	\[
		\tau_i = \{mt(ms^{-1}(v)) \mapsto mt(ms^{-1}(\rho_i(v))) \st v \in \dom \rho_i \}
	\]
	
	Because $TGG$ is PC, it holds that $mt(ms^{-1}(v)) \in H^i_t$ and $mt(ms^{-1}(\rho_i(v))) \in H^i_t$ for all $v$ in $\dom \rho_i$, thus $\tau_i$ is well defined. Moreover, the induction hypothesis supports that
	\[
		\forall v \in \dom \rho_i \. \rho_i(v) \notin \dom \rho_i \text{ and } \phi_{H^i_s}(\rho_i(v)) = \phi_{H^i_s}(v)
	\]
	Hence, by $mt$ and $ms$ bijectivity and by the PC property, it is clear that
	\[
		\forall v \in \dom \tau_i \. \tau_i(v) \notin \dom \tau_i \text{ and } \phi_{H^i_t}(\tau_i(v)) = \phi_{H^i_t}(v)
	\]
	Thus, choosing $V_{H_t}$ and $E_{H_t}$ according to Definition \ref{def:tresolv} and $\tau_i$, we have that if $H^i_s \tresolv{\rho_{i}} H^{i+1}_s$ then $(H^i_s \ms{ms} H^i_c \mt{mt} H^i_t) \tresolv{\rho_i,\tau_i} (H^{i+1}_s \ms{} H^{i+1}_c \mt{} H^{i+1}_t)$.
	
	This finishes the proof.
\end{proof}

The effective transformation procedure for TGG with PAC is essentially the same as the one in Algorithm \ref{alg:transform}, with the additional requirement of TGG being PC and the use of Theorem \ref{thm:one_d_enough_pac} to derive and resolve PAC vertices for the produced triple graph.