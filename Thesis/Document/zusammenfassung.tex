\chapter*{Zusammenfassung}

Die vorliegende Masterarbeit stellt einen neuen Graphgrammatikformalismus vor, der das Konzept vom Nichtterminalsymbol in die Theorie von Tripel-Graph-Grammatik einbringt. Dieser neue Formalismus ergibt sich aus der Kombination von \emph{Graph Grammars With Neighborhood-Controlled Embedding} (NCE Graphgrammatiken) und \emph{Tripel-Graph-Grammatiken} (TGG) und wird in dieser Arbeit NCE TGG genannt.

Normalerweise beziehen sich TGG-basierte Ansätze auf monotone, kontextsensitive Tripel-Graph-Grammatiken. Im Gegensatz dazu bezieht sich der NCE TGG Ansatz auf kontextfreie Tripel-Graph-Grammatiken, der eine Nutzung von Nichtterminalsymbolen als Labels für Knoten eines Graphen erlaubt. Ein solcher Unterschied führt dazu, dass die mit NCE TGG spezifizierten Modelltransformationen im Durchschnitt weniger Regeln enthalten, und bieten dem Nutzer einen besseren Mechanismus für die Darstellung von abstrakten Konzepten in Modelltransformationen an.

In der vorliegenden Arbeit wird NCE TGG definiert und eine Methode für Modelltransformationen vorgestellt, die auf einem Parser-Algorithmus für NCE Graphgrammatiken basiert. Darüber hinaus wird eine Erweiterung von NCE TGG präsentiert, die \emph{Application Conditions} unterstützt und die generative Kraft verbessern soll. Für die Erweiterung wird sowohl ein Parser-Algorithmus, als auch eine Modelltransformationsmethode eingeführt. Des Weiteren werden eine Diskussion zur Implementierung, sowie zwei Fallstudien von NCE TGG und Modelltransformationen dargelegt.

Das Ergebnis der experimentellen Evaluation übertrifft, in zwei von fünf Fällen und auf den Gesamtdurchschnitt gesehen, die Nutzbarkeit von NCE TGG für Modelltransformationen. Daraus lässt sich erkennen, dass dieser Mechanismus für die Darstellung von abstrakten Konzepten in einer Graphgrammatik die Nutzbarkeit erhöhen kann.