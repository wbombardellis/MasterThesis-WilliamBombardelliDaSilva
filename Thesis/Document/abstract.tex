\chapter*{Abstract}

This thesis presents a novel graph grammar formalism that introduces the concept of non-terminal symbols to the already existent theory of triple graph grammars. This new formalism is the result of the mixture of \emph{graph grammars with neighborhood-controlled embedding} (NCE graph grammars) with \emph{triple graph grammars} (TGG) and we name it NCE TGG.

Differently from other TGG-based approaches for model transformation, that are based on monotonic context-sensitive triple graph grammars, the NCE TGG approach is based on context-free triple graph grammars that allow the labeling of graph vertices with non-terminal symbols. Such difference makes model transformations specified with the latter, in average, shorter and provides the user with a better mechanism for the representation of abstract concepts in model transformations.

In this work, we define NCE TGG and present a model transformation method for it, based on a parsing algorithm for NCE graph grammars. Furthermore, aiming at the increase of generative power of NCE TGG, we put forward an extension of it---that supports application conditions---and present both a parsing algorithm and a model transformation method for it. An implementation discussion, as well as two case studies demonstrating the use of NCE TGG for the solving of the model transformation problem, are also expounded in this thesis.

An experimental evaluation on the usability of NCE TGG for model transformation reveals the potential of our approach since it outperforms the standard version of TGG in two out of five evaluated cases and in average. This shall evince how a good mechanism for the representation of abstract concepts in a graph grammar enhances its usability.