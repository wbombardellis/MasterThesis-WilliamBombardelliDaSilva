%%%%%%%%%%%%%%%%%%%%%%%%%%%%%%%%%%%%%%%%%%%%%%%%%%%%%%%%%%%%%%%%%%%%%%%%
%%%%%%%%%%%%%%%%%%%%%%%%%%%%%% CONCLUSION %%%%%%%%%%%%%%%%%%%%%%%%%%%%%%
%%%%%%%%%%%%%%%%%%%%%%%%%%%%%%%%%%%%%%%%%%%%%%%%%%%%%%%%%%%%%%%%%%%%%%%%

Summary and closing words. Outlook. Future work (e.g. lexicalization for model synchronization).

%TODO cite: s_52,s_93

%TODO: Enhancement: reduction of search space to avoid useless computation, and faster computation
%TODO: Enhancement: left hand side more flexible with three different symbols (e.g. A-B-C -> ...)

%TODO: Say it shed a light in the theory with practical discussion (special parsing)

%TODO: Talk about transformation in which a source graph could generate more than one target graph... Non-functional transformation
%TODO: Say that embeddings are complicated

%TODO: Expressive enough to model non connected graphs, unrooted, with cycles

%TODO: Future work: more theoretical results concersing e.g. expressiveness
%TODO: Future work: we call for empyrical analysis to evaluate usability not in terms of siye but in terms of perceived usability by modelers - Lack of context may be perceived negatively by some
%TODO: Embeddings may also be perceived as too complicated. Difficult to imagine what happens for different neighborhoods. I.e. uncertainty about neighborhod. Pure context match could be better -> HRG
%TODO: Say downside: Fewer theoretical results as TGG. It has stronger theory (category theory) behind it.
%TODO: Say this evaluation measures only one aspect of the usability (size), we wished to evaluate how a modeller would work with it -> empyrism
