%%%%%%%%%%%%%%%%%%%%%%%%%%%%%%%%%%%%%%%%%%%%%%%%%%%%%%%%%%%%%%%%%%%%%%%%
%%%%%%%%%%%%%%%%%%%%%%%%%%% INTRODUCTION %%%%%%%%%%%%%%%%%%%%%%%%%%%%%%%
%%%%%%%%%%%%%%%%%%%%%%%%%%%%%%%%%%%%%%%%%%%%%%%%%%%%%%%%%%%%%%%%%%%%%%%%

Overview of the MDE. Potential and problems of it. The challenge of model transformation. A solution: Triple Graph Grammars (TGG) (justification). A problem of TGG (usability/ amount of grammar rules). Our solution for this problem: TGG with non-terminal nodes. Overview of our approach (graph grammars with non-terminal nodes, NCE grammars, graph language, parsing, transformation). Brief intro into graph grammars and embedding. Explain why BNCE (HRG parsers were for classes too restricted), why grammars. Short summary of the results. Remainder.

%TODO: Reread intros of the two papers and take good things from there
% Our proposal is inspired by the use non-terminal symbols in string grammars, that allows the creation of simpler grammars for which there are more efficient recognizer algorithms and with which it has become possible to built efficient programming language compilers.