%%%%%%%%%%%%%%%%%%%%%%%%%%%%%%%%%%%%%%%%%%%%%%%%%%%%%%%%%%%%%%%%%%%%%%%%
%%%%%%%%%%%%%%%%%%%%%%%% MODEL TRANSFORMATION %%%%%%%%%%%%%%%%%%%%%%%%%%
%%%%%%%%%%%%%%%%%%%%%%%%%%%%%%%%%%%%%%%%%%%%%%%%%%%%%%%%%%%%%%%%%%%%%%%%
As already introduced, TGG can be used to characterize languages of triple graphs holding correctly transformed models. That is, one can interpret a TGG as the description of the correctly-transformed relation between two sets of models $\mathcal{S}$ and $\mathcal{T}$, where two models $G \in \mathcal{S}$ and $T \in \mathcal{T}$ are in the relation if, and only if, $G$ and $T$ are respectively, source and target graphs of any triple graph of the language $L(TGG)$. That being said, we are interested in this chapter on defining a model transformation algorithm that interprets a BNCE TGG to transform a source model $G$ into one of its correspondent target models $T$ according to the correctly-transformed relation defined by the TGG.

For that end, let $TGG = (\Sigma = \Sigma_s \cup \Sigma_t, \Delta, S, P)$ be a triple graph grammar defining the correctly-transformed relation between two arbitrary sets of graphs $\mathcal{S}$ over $\Sigma_s$ and $\mathcal{T}$ over $\Sigma_t$. And let $G \in \mathcal{S}$ be a source graph. We want to find a target graph $T \in \mathcal{T}$ such that $G \ms{} C \mt{} T \in L(TGG)$. To put in words, we wish to find a triple graph holding $G$ and $T$ that is in the language of all correctly transformed models. Hence, the model transformation problem is reduced--- according to the definition of triple graph language (see Definition \ref{def:tlanguage})--- to the problem of finding a derivation $\startTG{TGG} \tderivtr{TGG} G \ms{} C \mt{} T$.

Our strategy to solve this problem is, first, to get a derivation for $G$ with the source part of $TGG$ and, then, construct the derivation $\startTG{TGG} \tderivtr{TGG} G \ms{} C \mt{} T$. For this purpose, consider the definitions of the $\source$ function, that extract the source part of production rules.

%TODO: Maybe add an image summarizing the approach

\begin{definition}
	\label{def:source}
	Let $r = (A \pro (G_s \ms{} G_c \mt{} G_t), \omega_s, \omega_t)$ be a production rule of a triple graph grammar, $\source(r) = (A \pro G_s,\omega_s)$ gives the source part of $r$ and $\source^{-1}((A \pro G_s,\omega_s)) = r$ is the inverse of $\source(r)$.
\end{definition}

In order for $s^{-1}$ to be well defined, we require that all source parts $(A \pro G_s,\omega_s)$ be unique. This does not affect the generality of the formalism, for isomorphic right-hand side graphs are still allowed.

\begin{definition}
	\label{def:Source}
	Let $TGG = (\Sigma, \Delta, S, P)$ be a triple graph grammar, $\Source(TGG) = (\Sigma, \Delta, S, \source(P))$ gives the source grammar of $TGG$.
\end{definition}

Furthermore, consider the definition of the non-terminal consistent (NTC) property of TGG, which assures that non-terminal vertices of the correspondent graph are connected to vertices with the same label in the source and target graphs.

\begin{definition}
	A triple graph grammar $TGG = (\Sigma, \Delta, S, P)$ is \emph{non-terminal consistent} (NTC) if, and only if, for all rules $(A \pro (G_s \ms{ms} G_c \mt{mt} G_t), \omega_s, \omega_t) \in P$, the following holds
	\begin{equation*}
		\forall c \in V_{G_c} \. \text{if } \phi_{G_c}(c) \in \Gamma \text{ then } \phi_{G_c}(c) = \phi_{G_s}(ms(c)) = \phi_{G_t}(mt(c))
	\end{equation*}
\end{definition}

Finally, the following result gives us an equivalence between a derivation in $TGG$ and a derivation in its source grammar $S(TGG)$, which allows us to construct our goal derivation of $G \ms{} C \mt{} T$ in $TGG$ using the derivation of $G$ in $S(TGG)$.

\begin{theorem}
	\label{thm:one_d_enough}
	Let $TGG = (\Sigma, \Delta, S, P)$ be a NTC NCE TGG and $k \ge 1$, \\
	$D = (\startTG{TGG} \tderiv{r_0}{s_0,c_0,t_0}{} G^1 \tderiv{r_1}{s_1,c_1,t_1}{} \dots \tderiv{r_{k-1}}{s_{k-1},c_{k-1},t_{k-1}}{} G^k)$ is a derivation in $TGG$ if, and only if, $\overline{D} = (\startG{S(TGG)} \deriv{s(r_0)}{s_0}{} G^1_s \deriv{s(r_1)}{s_1}{} \dots \deriv{s(r_{k-1})}{s_{k-1}}{} G^k_s)$ is a derivation in $S(TGG)$.
\end{theorem}
\begin{proof}
	We want to show that if $D$ is a derivation in $TGG = (\Sigma, \Delta, S, P)$, then $\overline{D}$ is a derivation in $SG := S(TGG) = (\Sigma, \Delta, S, SP)$, and vice-versa. We prove it by induction in the following.
	
	First, for the induction base, since, $\startTG{TGG} \tderiv{r_0}{s_0,c_0,t_0}{TGG} G^1$, then expanding $\startTG{TGG}$ and $G^1$, we have
	\begin{align*}
		& Z_s \ms{} Z_c \mt{} Z_t \tderiv{r_0}{s_0,c_0,t_0}{TGG} G^1_s \ms{} G^1_c \mt{} G^1_t \text{, then, by Definition \ref{def:tgg_dstep},}\\
		& r_0 = (S \pro (R_s \ms{} R_c \mt{} R_t), \omega_s, \omega_t) \in P \text{ and, by Definition \ref{def:source},}\\
		& \source(r_0) = (S \pro R_s, \omega_s) \in SP
	\end{align*}
	
	Hence, using it plus the configuration of $\phi_{Z_s}(s_0)$, $V_{G^1_s}$, $E_{G^1_s}$ and $\phi_{G^1_s}$ and the equality $Z_s = \startG{SG}$, we have, by Definition \ref{def:gg_dstep}, $\startG{SG} \deriv{\source(r_0)}{s_0}{SG} G^1_s$.
	
	In the other direction, we choose $c_0, t_0$ from the definition of $\startTG{TGG}$, with $\phi_{Z_c}(c_0) = S$ and  $\phi_{Z_t}(t_0) = S$. In this case, since,
	\begin{align*}
		& \startG{SG} \deriv{s(r_0)}{s_0}{SG} G^1_s \text{, then by Definition \ref{def:gg_dstep},}\\
		& \source(r_0) = (S \pro R_s, \omega_s) \in SP \text{ and, using the bijectivity of $\source$, we get}\\
		& r_0 = \source^{-1}(s(r_0)) = (S \pro (R_s \ms{} R_c \mt{} R_t), \omega_s, \omega_t) \in P
	\end{align*}
	
	Hence, using it plus the configuration of $\phi_{\startG{SG}}(s_0)$, $V_{G^1_s}$, $E_{G^1_s}$ and $\phi_{G^1_s}$, the equality $Z_s = \startG{SG}$ and constructing $V_{G^1_c}$, $V_{G^1_t}$, $E_{G^1_c}$, $E_{G^1_t}$, $\phi_{G^1_c}$, $\phi_{G^1_t}$ from $Z_c$ and $Z_t$ according to the Definition \ref{def:tgg_dstep}, we have $\startTG{TGG} \tderiv{r_0}{s_0,c_0,t_0}{TGG} G^1_s \ms{} G^1_c \mt{} G^1_t$.
	
	Now, for the induction step, we want to show that if $\startTG{TGG} \tderivtr{TGG} G^i$ $\tderiv{r_{i}}{s_{i},c_{i},t_{i}}{TGG} G^{i+1}$ is a derivation in $TGG$, then $\startG{SG} \derivtr{SG} G^i_s \deriv{s(r_{i})}{s_{i}}{SG} G^{i+1}_s$ is a derivation in $SG$ and vice-versa, provided that the equivalence holds for the first $i$ steps. Then, we just have to show it for the step $i+1$.
	
	So, since, $G^i \tderiv{r_{i}}{s_{i},c_{i},t_{i}}{TGG} G^{i+1}$, that is
	\begin{align*}
		& G^i_s \ms{ms_i} G^i_c \mt{mt_i} G^i_t \tderiv{r_i}{s_i,c_i,t_i}{TGG} G^{i+1}_s \ms{} G^{i+1}_c \mt{} G^{i+1}_t \text{, then, by Definition \ref{def:tgg_dstep},}\\
		& r_i = (S \pro (R_s \ms{} R_c \mt{} R_t), \omega_s, \omega_t) \in P \text{, and by Definition \ref{def:source},}\\
		& \source(r_i) = (S \pro R_s, \omega_s) \in SP
	\end{align*}
	
	Hence, using it plus the configuration of $\phi_{G^i_s}(s_i)$, $V_{G^{i+1}_s}$, $E_{G^{i+1}_s}$ and $\phi_{G^{i+1}_s}$, we have, by Definition \ref{def:gg_dstep}, $G_s^i \deriv{\source(r_i)}{s_i}{SG} G^{i+1}_s$.
	
	In the other direction, we choose, using the NTC property, $c_i = ms_i^{-1}(s_i), t_i = mt_i(c_i)$. Moreover, since $TGG$ is NTC, and because, by induction hypothesis, $\startTG{TGG} \tderivtr{TGG} G^i$ is a derivation in $TGG$ and $\phi_{G^i_s}(s_i) \in \Gamma$, it is clear that $\phi_{G^i_s}(s_i) = \phi_{G^i_c}(c_i) = \phi_{G^i_t}(t_i)$.
	
	In this case, since
	\begin{align*}
		& G^{i}_s \deriv{s(r_i)}{s_i}{SG} G^{i+1}_s \text{, then, by Definition \ref{def:gg_dstep},}\\
		& \source(r_i) = (A \pro R_s, \omega_s) \in SP \text{ and, using the bijectivity of $\source$, we get }\\
		& r_i = \source^{-1}(s(r_i)) = (A \pro (R_s \ms{} R_c \mt{} R_t), \omega_s, \omega_t) \in P
	\end{align*} 
	
	Hence, using, additionally, the configuration of $\phi_{G^i_s}(s_i)$, $\phi_{G^i_c}(c_i)$, $\phi_{G^i_t}(t_i)$, $V_{G^{i+1}_s}$, $E_{G^{i+1}_s}$ and $\phi_{G^{i+1}_s}$ and constructing $V_{G^{i+1}_c}$, $V_{G^{i+1}_t}$, $E_{G^{i+1}_c}$, $E_{G^{i+1}_t}$, $\phi_{G^{i+1}_c}$, $\phi_{G^{i+1}_t}$ from $G^i_c$ and $G^i_t$ according to the Definition \ref{def:tgg_dstep}, we have 
	\begin{equation*}
		G^i_s \ms{} G^i_c \mt{} G^i_t \tderiv{r_i}{s_i,c_i,t_i}{TGG} G^{i+1}_s \ms{} G^{i+1}_c \mt{} G^{i+1}_t
	\end{equation*}
	
	This finishes the proof.
\end{proof}

Therefore, by Theorem \ref{thm:one_d_enough}, the problem of finding a derivation $D = (\startTG{TGG} \tderivtr{TGG} G \ms{} C \mt{} T)$ is reduced to finding a derivation $\overline{D} = (\startG{S(TGG)} \deriv{}{}{S(TGG)} G)$, what can be done with the already presented parsing Algorithm \ref{alg:parse}. The final construction of the triple graph $G \ms{} C \mt{} T$ becomes then just a matter of creating $D$ out of $\overline{D}$.

The complete transformation procedure is presented in the Algorithm \ref{alg:transform}. Thereby, it is required that the TGG be neighborhood preserving (NP), what poses no problem to our procedure, once any TGG can be transformed into a neighborhood preserving normal form. A more detailed discussion of the NP normalization is performed in Chapter \ref{ch:Implementation}. Also notice that this algorithm always terminates. A more elaborated discussion on its complexity is given in Section \ref{sec:eval-performance}.

\begin{algorithm}[!h]
	\caption{Transformation Algorithm for NP NTC BNCE TGG}
	\begin{algorithmic}[!ht]
		\Require $TGG \text{ is a valid NP NTC BNCE triple graph grammar}$
		\Require $G \text{ is a valid graph over } \Delta$
		\Function{$transform$}{$TGG=(\Sigma, \Delta, S, P), G=(V_G,E_G,\phi_G)$}{$:Graph$}
		\State $SG \gets S(TGG)$ \Comment{see Definition \ref{def:source}}
		\State $\overline{D} \gets parse(SG,G)$ \Comment{use Algorithm \ref{alg:parse}}
		\If{$\overline{D} = (\startG{SG} \derivtr{SG} G)$} \Comment{if parsed successfully}
		\State from $\overline{D}$ construct $D = (\startTG{TGG} \tderivtr{TGG} G \ms{} C \mt{} T)$
		\State \Return {$\Just T$}
		\Else
		\State \Return {$\Nothing$} \Comment{no $T$ satisfies $(G \ms{} C \mt{} T) \in L(TGG)$}
		\EndIf
		\EndFunction 
		\Ensure $return \text{ is either } \Nothing \text{ or } \Just T \text{, such that } (G \ms{} C \mt{} T) \in L(TGG)$
	\end{algorithmic}
	\label{alg:transform}
\end{algorithm}

Our transformation method is robust enough to support both ambiguous grammars, because the parser supports it, and non-functional transformations. A transformation specified by the triple graph grammar $TGG$ is non-functional if, and only if, for at least one source graph $G$ there is more than one target graph $T$, such that $G \ms{} C \mt{} T \in L(TGG)$ \cite{Hermann2010}. This can happen when the source part's right-hand side of two rules are isomorphic but their target parts not. An evidence for a transformation to be non-functional is the existence of two rules $r$ and $q$ in the set of rules of $TGG$ with $\source(r) = (A \pro R_r,\omega_r)$, $\source(q) = (B \pro R_q,\omega_q)$ and $R_r \isomorph R_q$. In this case, the output for the source graph $G$ is non-deterministic, because any of the rules $r$ or $q$ may be chosen by the parser at the construction of the parsing tree for $G$.

We claim, moreover, that our transformation method is correct and complete in the TGG sense \cite{Hermann2013}. The former shall hold, because all the set of triple graphs generated by our algorithm is contained by $L(TGG)$ for every NP NTC BNCE TGG. The latter shall hold, because $L(TGG)$ is contained by the set of all triple graphs generated by our algorithm for every NP NTC BNCE TGG.
