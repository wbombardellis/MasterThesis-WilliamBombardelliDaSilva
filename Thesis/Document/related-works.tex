%%%%%%%%%%%%%%%%%%%%%%%%%%%%%%%%%%%%%%%%%%%%%%%%%%%%%%%%%%%%%%%%%%%%%%%%
%%%%%%%%%%%%%%%%%%%%%%%%%%% RELATED WORKS %%%%%%%%%%%%%%%%%%%%%%%%%%%%%%
%%%%%%%%%%%%%%%%%%%%%%%%%%%%%%%%%%%%%%%%%%%%%%%%%%%%%%%%%%%%%%%%%%%%%%%%

In this section, we offer a literary review on the topics of graph grammars and triple graph grammars as well as we indicate published works that are related with our approach. Here, we focus on the node label and the hyperedge replacement approach for graph grammars. Nevertheless, the field does not restrict to this topic, instead, there is a myriad of different approaches to it, for example, the algebraic approach \cite{ehrig1999handbook}. We refer to context-free and context-sensitive grammars, inspired by the use of such classification for string grammars, in a relaxed way without any compromise to the correct definition of context-freeness for graph grammars.

\textit{Hyperedge replacement graph grammars} (HRG) are context-free grammars with semantics based on the replacement of hyperedges by hypergraphs \cite{drewes1997hyperedge} governed by morphisms. Prominent polynomial-time top-down and shift-reduce parsing techniques for classes of such grammars can be found in \cite{drewes2015predictive,drewes2017predictive,bjorklund2016between,chiang2013parsing} and applications for syntax definition of a visual language can be found in \cite{minas2006syntax,engelfriet1998tree}.

We divide the node label replacement approaches into context-sensitive and context-free approaches, we refer to context-sensitive and context-free grammars, inspired by the use of such classification for string grammars, in a relaxed way without any compromise to any definition of context-freeness for graph grammars. The context-sensitive field includes the \textit{layered graph grammar}, whose semantics consists of the replacement of graphs by other graphs governed by morphisms \cite{rekers1997defining} and for which exponential-time bottom-up parsing algorithms have been proposed \cite{rekers1995graph,bottoni2000efficient,furst2011improving}. Another context-sensitive formalism is the \textit{reserved graph grammar}, that is based on the replacement of directed graphs by necessarily greater directed graphs governed by simple embedding rules \cite{zhang2001context} and for which exponential and polynomial-time bottom-up algorithms have been proposed in \cite{zeng2005rgg+,zou2017partial}.

In the node label replacement context-free formalisms stand out the  \textit{node label controlled graph grammar} (NLC) and its successor \textit{graph grammar with neighborhood-controlled embedding} (NCE). NLC is based on the replacement of one vertex by a graph, governed by embedding rules written in terms of the vertex's label \cite{rozenberg1986boundary}. For various classes of these grammars, there exists polynomial-time top-down and bottom-up parsing algorithms \cite{flasinski1993parsing,flasinski2014characteristics, rozenberg1986boundary, wanke1991algorithms}. The recognition complexity and generation power of such grammars have also been analyzed \cite{flasinski1998power,kim2012structure}. NCE occurs in several formulations, including a context-sensitive one, but here we focus on the context-free formulation, where one vertex is replaced by a graph, and the embedding rules are written in terms of the vertex's neighbors \cite{janssens1982graph,skodinis1998neighborhood}. For some classes of these grammars, polynomial-time bottom-up parsing algorithms and automaton formalisms were proposed and analyzed \cite{kim2001efficient,brandenburg2005finite}. In special, one of these classes is the \textit{boundary graph grammar with neighborhood-controlled embedding} (BNCE), that is used to construct our own formalism. Moreover, it is worth mentioning that, according to \cite{engelfiet1990comparison}, BNCE and HRG have the same generative power.

%TODO: "Generating Efficient Parsers for HR grammars", Cite s_78 for evaluation of HRGs
%TODO: HR are too few expressive

Beyond the approaches presented above, there is a myriad of alternative proposals for graph grammars, including a context-sensitive NCE \cite{adachi1999nce}, an edge-based grammar \cite{shi2015method}, a grammar that replaces star graphs by other graphs \cite{drewes2010adaptive}, a coordinate system-based grammar \cite{kong2006spatial} and a regular graph grammar \cite{gilroy2017parsing}.

Regarding TGG \cite{schurr1994specification}, a 20 years review of the realm is put forward by Anjorin et al. \cite{anjorin201620}. In special, advances are made in the direction of expressiveness with the introduction of application conditions \cite{klar2010extended} and of modularization \cite{anjorin2014modularizing}. Furthermore, in the algebraic approach for graph grammars, we have found proposals that introduce inheritance \cite{bardohl2004integrating,hermann2008typed} and variables \cite{hoffmann2005graph} to the formalisms. Nevertheless, we do not know any approach that introduces non-terminal symbols to TGG with the purpose of gaining expressiveness or usability. In this sense our proposal brings something new to the current state-of-the-art.

%TODO: Add briefly review on other model transformation approaches: QVT, ATL,...