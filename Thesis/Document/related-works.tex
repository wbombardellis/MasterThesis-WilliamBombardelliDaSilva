%%%%%%%%%%%%%%%%%%%%%%%%%%%%%%%%%%%%%%%%%%%%%%%%%%%%%%%%%%%%%%%%%%%%%%%%
%%%%%%%%%%%%%%%%%%%%%%%%%%% RELATED WORKS %%%%%%%%%%%%%%%%%%%%%%%%%%%%%%
%%%%%%%%%%%%%%%%%%%%%%%%%%%%%%%%%%%%%%%%%%%%%%%%%%%%%%%%%%%%%%%%%%%%%%%%
In this chapter, we offer a literary review on the topics of model transformation and graph grammars as well as we indicate published works that are related to our approach. In the process, we split model transformation approaches into operational and relational and graph grammar approaches into algebraic, hyperedge-based and node label-based. Lastly, we give an overview on the literature of triple graph grammars with a special focus on proposals that relate to our work.

Mens and Van Gorp \cite{mens2006taxonomy} and Czarnecki and Helsen \cite{czarnecki2003classification} offer taxonomies of model transformation that put several variations of the problem and different techniques to solve it in different frameworks. Thereby, two categories of model transformation approaches stand out. The operational (also named direct-manipulation) approach aims at describing transformation in an imperative manner, in which the user writes algorithms for each desired transformation. One language for the writing of such algorithms is the QVT-O \cite{omg2008meta}. But the approach in which we are more interested in this thesis is the relational approach, that aims at describing transformation in terms of transformation rules. Two popular technologies that address this strategy are the QVT-R \cite{omg2008meta} and the ATL \cite{jouault2006atl}, which offer praxis-oriented languages and environments for the development and execution of model transformations.

These practical methods are very valuable for the model transformation realm, but for the development of this thesis we explore more the theoretical methods for it. In especial, we are interested in the theory of graph grammars, which are, indeed, the basis for the implementation of the aforementioned technologies. Ehrig et al. \cite{ehrig1999handbook} provide a review on the different graph grammar approaches, including the \emph{algebraic approach}, in which category theory concepts are used to express graph transformations. In particular, pushout diagrams involving morphisms between grammar rules and graph instances are used to depict rule applications \cite{corradini1997algebraic}. Interestingly, in the double pushout approach, rules are not of the form $L \pro R$, instead they are of the form $L \pre K \pro R$, where $K$ gives the vertices that are maintained after a rule application.

Another kind of graph grammars are the \textit{hyperedge replacement graph grammars} (HRG), which are context-free grammars with semantics based on the replacement of hyperedges by hypergraphs \cite{drewes1997hyperedge} governed by morphisms. That is to say, HRG rules are of the type $L \pro R$, where $L$ is a hyperedge and $R$ a hypergraph. Prominent polynomial-time top-down and shift-reduce parsing techniques for classes of such grammars can be found in Drewes et al. \cite{drewes2015predictive, drewes2017predictive}, Bj\"{o}rklund \cite{bjorklund2016between} and Chiang et al. \cite{chiang2013parsing} and applications for syntax definition of a visual language can be found in Minas \cite{minas2006syntax} and Engelfriet and Maneth \cite{engelfriet1998tree}. We refer to context-free and context-sensitive grammars, inspired by the use of such classification for string grammars, in a relaxed way without any compromise to the correct definition of context-freeness for graph grammars.

We divide the node label replacement approaches into context-sensitive and context-free approaches. The context-sensitive field includes the \textit{layered graph grammar}, whose semantics consists of the replacement of graphs by other graphs governed by morphisms \cite{rekers1997defining} and for which exponential-time bottom-up parsing algorithms have been proposed \cite{rekers1995graph,bottoni2000efficient,furst2011improving}, in this case, rules are of the form $L \pro R$, where $L$ is a graph and $R$ another graph. Another context-sensitive formalism is the \textit{reserved graph grammar}, that is based on the replacement of directed graphs by necessarily greater directed graphs (i.e. $L \subset R$) governed by simple embedding rules \cite{zhang2001context} and for which exponential and polynomial-time bottom-up algorithms have also been proposed \cite{zeng2005rgg+,zou2017partial}.

In the node label replacement context-free formalisms stand out the  \textit{node label controlled graph grammar} (NLC) and its successor \textit{graph grammar with neighborhood-controlled embedding} (NCE). NLC is based on the replacement of one vertex by a graph, governed by embedding rules written in terms of the vertex's label \cite{rozenberg1986boundary}, in this case, rules are of the form $L \pro R$, where $L$ is a label and $R$ is a graph. For various classes of these grammars, there exist polynomial-time top-down and bottom-up parsing algorithms \cite{flasinski1993parsing,flasinski2014characteristics, rozenberg1986boundary, wanke1991algorithms}. The recognition complexity and generation power of such grammars have also been analyzed \cite{flasinski1998power,kim2012structure}. NCE occurs in several formulations, including a context-sensitive one, but here we focus on the context-free formulation, where one vertex is replaced by a graph, and the embedding rules are written in terms of the vertex's neighbors \cite{janssens1982graph,skodinis1998neighborhood}. For some classes of these grammars, polynomial-time bottom-up parsing algorithms and automaton formalisms were proposed and analyzed \cite{kim2001efficient,brandenburg2005finite}. In special, one of these classes is the \textit{boundary graph grammar with neighborhood-controlled embedding} (BNCE), which is used to construct our own formalism. Moreover, it is worth mentioning that, according to Engelfriet and Rozenberg \cite{engelfiet1990comparison}, BNCE and HRG have the same generative power. Nevertheless, we opt for BNCE because the efficient parsers for HRG \cite{drewes2015predictive, drewes2017predictive} work only for a class of grammars that are very restricted for our goals.

Beyond the approaches presented above, there is a myriad of alternative proposals for graph grammars, including a context-sensitive NCE \cite{adachi1999nce}, an edge-based grammar \cite{shi2015method}, a grammar that replaces star graphs by other graphs \cite{drewes2010adaptive}, a coordinate system-based grammar \cite{kong2006spatial} and a regular graph grammar \cite{gilroy2017parsing}.

Regarding TGG \cite{schurr1994specification}, a 20 years review of the realm is put forward by Anjorin et al. \cite{anjorin201620}. In special, advances are made in the direction of expressiveness with the introduction of application conditions \cite{klar2010extended} and of modularization \cite{anjorin2014modularizing}. Furthermore, in the algebraic approach for graph grammars, we have found proposals that introduce inheritance \cite{bardohl2004integrating,hermann2008typed} and variables \cite{hoffmann2005graph} to the formalisms. Nevertheless, we do not know any approach that introduces non-terminal symbols to TGG with the purpose of gaining expressiveness or usability. In this sense our proposal brings something new to the current state-of-the-art.