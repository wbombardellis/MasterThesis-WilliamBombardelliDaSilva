%%%%%%%%%%%%%% THEOREM %%%%%%%%%%%%%%
\theoremstyle{definition}
\newtheorem*{definition}{Definition}
\newtheorem*{example}{Example}

\theoremstyle{plain}
\newtheorem{theorem}{Theorem}

\theoremstyle{remark}
\newtheorem*{remark}{Remark}

%%%%%%%%%%%%%% GENERAL %%%%%%%%%%%%%%
\newcommand{\rel}{\sim}
\newcommand{\rrestr}{\triangleright}
\newcommand{\st}{\; | \;}
\renewcommand{\.}{. \;}
\newcommand{\dom}{\operatorname{dom}}

%%%%%%%%%%%%%% GG %%%%%%%%%%%%%%
\newcommand{\neigh}[1]{\operatorname{neigh}_{#1}}
\newcommand{\isomorph}{\cong}
\newcommand{\scont}[2]{\eta_{#1}(#2)}
\newcommand{\cont}[2]{\operatorname{cont}_{#1}(#2)}
\newcommand{\emb}[1]{\operatorname{emb}(#1)}
\newcommand{\allgraphs}[1]{\mathcal{G}_{#1}}
\newcommand{\startG}[1]{Z_{#1}}
\newcommand{\pro}{\to}

%%%%%%%%%%%%%% TGG %%%%%%%%%%%%%%
\newcommand{\alltgraphs}[1]{\mathcal{TG}_{#1}}
\newcommand{\startTG}[1]{Z_{#1}}
\newcommand{\ms}[1]{\overset{#1}{\leftarrow}}
\newcommand{\mt}[1]{\overset{#1}{\rightarrow}}
\newcommand{\emptyTG}{\varepsilon}
\newcommand{\source}{\operatorname{s}}
\newcommand{\Source}{\operatorname{S}}
\newcommand{\target}{\operatorname{t}}
\newcommand{\Target}{\operatorname{T}}
\newcommand{\cderiv}[3]{
	\ifx\relax#2\relax%
	\overset{#1}{\Rrightarrow}_{#3}%
	\else
	\overset{#1,#2}{\Rrightarrow}_{#3}%
	\fi
}
\newcommand{\deriv}[3]{
	\ifx\relax#2\relax
	\overset{#1}{\Rightarrow}_{#3}
	\else
	\overset{#1,#2}{\Rightarrow}_{#3}
	\fi
}
\newcommand{\derivtr}[1]{
	\Rightarrow^*_{#1}
}
\newcommand{\tcderiv}[3]{
	\ifx\relax#2\relax
	\overset{#1}{\Rrightarrow}_{#3}
	\else
	\overset{#1,#2}{\Rrightarrow}_{#3}
	\fi
}
\newcommand{\tderiv}[3]{
	\ifx\relax#2\relax
	\overset{#1}{\Rightarrow}_{#3}
	\else
	\overset{#1,#2}{\Rightarrow}_{#3}
	\fi
}
\newcommand{\tderivtr}[1]{\Rightarrow^*_{#1}}
\newcommand{\cderivpac}[4]{\overset{#1,#2,#3}{\Rrightarrow}_{#4}}
\newcommand{\derivpac}[4]{\overset{#1,#2,#3}{\Rightarrow}_{#4}}
\newcommand{\derivpacn}[2]{\Rightarrow^{#2}_{#1}}
\newcommand{\resolv}[1]{\overset{#1}{\rightarrowtail}}
\newcommand{\resolvn}[2]{\overset{#1}{\rightarrowtail}^{#2}}
\newcommand{\derivpactr}[1]{\Rightarrow^*_{#1}}
\newcommand{\resolvtr}[1]{\rightarrowtail^*}

\newcommand{\tcderivpac}[4]{\overset{#1,#2,#3}{\Rrightarrow}_{#4}}
\newcommand{\tderivpac}[4]{\overset{#1,#2,#3}{\Rightarrow}_{#4}}
\newcommand{\tderivpacn}[2]{\Rightarrow^{#2}_{#1}}
\newcommand{\tresolv}[1]{\overset{#1}{\rightarrowtail}}
\newcommand{\tresolvn}[2]{\overset{#1}{\rightarrowtail}^{#2}}
\newcommand{\tresolvtr}[1]{\Rightarrow^*_{#1}}

%%%%%%%%%%%%%% ALGORITHM %%%%%%%%%%%%%%
\newcommand{\?}{\;\operatorname{\textbf{?}}\;}
\renewcommand{\:}{\;\operatorname{\textbf{:}}\;}
\newcommand{\Break}{\textbf{break}}
\newcommand{\Select}{\textbf{select}}
\newcommand{\Program}{\textbf{program}\;}
\newcommand{\Just}{\operatorname{\text{Just}}\;}
\newcommand{\Nothing}{\operatorname{\text{Nothing}}\;}
\newcommand{\pto}{\nrightarrow}
\newcommand{\ptree}[3]{
	\ifx&#2&%
	(#1 \rightrightarrows #3)
	\else
	(#1^#2 \rightrightarrows #3)
	\fi
}

%%%%%%%%%%%%%% TIKZ %%%%%%%%%%%%%%
\tikzstyle{grammar}=[shorten >= 1pt, ->, draw=black!50, framed, background rectangle/.style={draw, rounded corners}, font=\scriptsize ]
\tikzstyle{graph}=[shorten >= 1pt, ->, draw=black!50, font=\scriptsize]
\tikzstyle{rid} = [inner sep=3pt, align=left, anchor=east]
\tikzstyle{nont}=[rectangle, inner sep=3pt, draw, fill=white, minimum width=5pt]
\tikzstyle{t}=[circle, inner sep=1pt, draw, fill=white, minimum width=5pt]
\tikzstyle{pac}=[circle, inner sep=1pt, draw, dotted, fill=white, minimum width=5pt]
\tikzstyle{empty}=[font=\Large, fill=white]
\tikzstyle{g}=[inner sep=3pt, fill=white]
\tikzstyle{lhs}=[inner sep=1pt, fill=white]
\tikzstyle{w}=[circle, inner sep=1pt, below=8pt, draw, fill=white, font=\tiny]
\tikzstyle{uw}=[circle, inner sep=1pt, above=8pt, draw, fill=white, font=\tiny]
\tikzstyle{edge}=[->, thin, -latex]
\tikzstyle{pacedge}=[->, thin, -latex, dotted]
\tikzstyle{edgeLabel}=[midway, above]
\tikzstyle{vledgeLabel}=[midway, left]
\tikzstyle{vredgeLabel}=[midway, right]
\tikzstyle{vbedgeLabel}=[above=3pt]
\tikzstyle{wedge}=[-, thin]
\tikzstyle{biedge}=[-, thin]
\tikzstyle{pipe}=[-, thick]
\tikzstyle{morph}=[-, thin, dashed, -latex]

\newcommand{\ridX}{-0.3}
\newcommand{\ridY}{0.5}
\newcommand{\lhsX}{-0.3}
\newcommand{\pipeUY}{-0.5}
\newcommand{\pipeBY}{0.5}

