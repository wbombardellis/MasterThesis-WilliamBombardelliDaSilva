\renewcommand{\.}{. \;}
\theoremstyle{definition}
\newtheorem*{definition}{Definition}
\newtheorem*{example}{Example}

\theoremstyle{plain}
\newtheorem{theorem}{Theorem}

\theoremstyle{remark}
\newtheorem*{remark}{Remark}

\newcommand{\pto}{\nrightarrow}

\newcommand{\allgraphs}[1]{\mathcal{G}_{#1}}
\newcommand{\alltgraphs}[1]{\mathcal{TG}_{#1}}
\newcommand{\emptyGraph}{\varepsilon}
\newcommand{\pro}{\to}

\newcommand{\st}{\; | \;}
\newcommand{\isomorph}{\cong}
\newcommand{\ms}[1]{\overset{#1}{\leftarrow}}
\newcommand{\mt}[1]{\overset{#1}{\rightarrow}}
\newcommand{\?}{\operatorname{?}}
\newcommand{\cderiv}[3]{
	\ifx&#2&%
	\overset{#1}{\Rrightarrow}_{#3}
	\else
	\overset{#1,#2}{\Rrightarrow}_{#3}
	\fi
}
\newcommand{\deriv}[3]{
	\ifx&#2&%
	\overset{#1}{\Rightarrow}_{#3}
	\else
	\overset{#1,#2}{\Rightarrow}_{#3}
	\fi
}
\newcommand{\derivtr}[3]{
	\ifx&#2&%
	\overset{#1}{\Rightarrow^*}_{#3}
	\else
	\overset{#1,#2}{\Rightarrow^*}_{#3}
	\fi
}
\newcommand{\tcderiv}[3]{
	\ifx&#2&%
	\overset{#1}{\Rrightarrow}_{#3}
	\else
	\overset{#1,#2}{\Rrightarrow}_{#3}
	\fi
}
\newcommand{\tderiv}[3]{
	\ifx&#2&%
	\overset{#1}{\Rightarrow}_{#3}
	\else
	\overset{#1,#2}{\Rightarrow}_{#3}
	\fi
}
\newcommand{\tderivtr}[3]{
	\ifx&#2&%
	\overset{#1}{\Rightarrow^*}_{#3}
	\else
	\overset{#1,#2}{\Rightarrow^*}_{#3}
	\fi
}

\tikzstyle{grammar}=[shorten >=1pt,->,draw=black!50]
\tikzstyle{graph}=[shorten >=1pt,->,draw=black!50]
\tikzstyle{rid} = [align=left, anchor=east]
\tikzstyle{nont}=[rectangle,inner sep=5pt, draw, fill=white, minimum width=20pt]
\tikzstyle{t}=[circle,inner sep=3pt, draw, fill=white, minimum width=15pt]
\tikzstyle{g}=[inner sep=3pt, fill=white]
\tikzstyle{lhs}=[inner sep=5pt, fill=white]
\tikzstyle{w}=[circle, inner sep=2pt, below=8pt, draw, fill=white]
\tikzstyle{edge}=[->, very thick, -latex]
\tikzstyle{edgeLabel}=[midway, above]
\tikzstyle{wedge}=[-, thick]
\tikzstyle{biedge}=[-, very thick]
\tikzstyle{pipe}=[-, very thick]
\tikzstyle{morph}=[-, very thick, dashed, -latex]
