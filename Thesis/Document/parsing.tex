\section{Parsing of Graphs with Graph Grammars}
In the last section we cleared how the concepts of graphs and languages fit together. In this section we are interested in the problem of deciding, given a BNCE graph grammar $GG$ and a graph $G$, whether $G \in L(GG)$. This is sometimes called the \textit{membership} problem and can be solved through a recognizer algorithm that always finishes answering yes if and only if $G \in L(GG)$ and no otherwise. A slight extension of this problem is the \textit{parsing} problem, which consists of deciding if $G \in L(GG)$ and finding a derivation $\startG{GG} \derivtr{} G$.

%TODO: The practical use of it: model checking... model validation

The parsing algorithm posed in this section is an imperative view of the method proposed by ()%TODO: cite 27
, which is basically a version fro graphs of the well-known CYK (Cocke-Young-Kassami) algorithm for parsing of strings with a context-free (string) grammar. Preliminarily to the actual algorithm's presentation, we introduce some necessary concepts that are used by it. The first of them is the neighborhood preserving normal form.

\begin{definition}
	\label{def:np}
	A BNCE graph grammar $GG = (\Sigma, \Delta, S, P)$ is neighborhood preserving (NP), if and only if, the embedding of each rule with left-hand side $A$ is greater or equal than the context of each $A$-labeled vertex in the grammar. That is, let 
	\[\cont{(A \pro R,\omega)}{v} =\{(l,\phi_{R}(w)) \st (v,l,w) \in E_R \text { or } (w,l,v) \in E_R \} \cup \omega(v) \]
	be the context of $v$ in the rule $(A \pro R,\omega)$ and
	\[\scont{GG}{A} = \bigcup_{(B \pro Q,\zeta) \in P, v \in V_Q, \phi_Q(v) = A} \cont{B \pro Q,\zeta}{v} \] 
	be the context of the symbol $A$ in the grammar $GG$, then $GG$ is a NP BNCE graph grammar, if and only if,
	\[ \forall r = (A \pro R,\omega) \in P \. \scont{GG}{A} \subseteq \bigcup_{v \ in V_R} \omega(v) \]
\end{definition}

The NP property is important to the correctness of the parsing algorithm. Furthermore, it is guaranteed that any BNCE graph grammar can be transformed in an equivalent NP BNCE graph grammar in polynomial time. More details in () %TODO: cite 27

The next paragraphs present zone vertices and zone graphs, that are our understanding of the concepts also from %TODO: cite 27

\begin{definition}
	\label{def:zv}
	A zone vertex $h$ of a graph $G$ over $\Sigma$ is a pair $(\sigma \in \Sigma, U \subseteq V_G)$, that is, a symbol from $\Sigma$ and a subset of the vertices of $G$.
	
	A zone vertex can be understood as a contraction of a subgraph of $G$ defined by the vertices $U$ into one vertex with symbol $\sigma$.
\end{definition}

\begin{definition}
	\label{def:z}
	Let $H = \{(\sigma_0,U_0),(\sigma_1,U_1),\dots,(\sigma_m,U_m)\}$ be a set of zone vertices of a graph $G$ over $\Sigma$ with disjoint vertices (i.e. $U_i \cap U_j = \emptyset$ for all $0 \leq i,j \leq m \text{ and } i \neq j$) and $V(H) = \bigcup_{0 \leq i \leq m}{U_i}$. A zone graph $Z(H)$ for $H$ is $Z(H) = (V, E, \phi)$ with $V$ being the zone vertices, $E \subseteq V \times \Sigma \times V$ the edges between zone vertices and $\phi: V \to \Sigma$ the labeling function, determined by
	\begin{align*}
		V & = H \cup \{(\phi_G(x),\{x\}) \st x \in \neigh{G}(V(H)) \}\\
		E & = \{((\sigma,U),l,(\eta,T)) \st (\sigma,U),(\eta,T) \in V \text{ and } U \neq T \text{ and } \\
		& (u,l,t) \in E_G \text{ and } u \in U \text{ and } t \in T\} \\
		\phi & = \{(\sigma,U) \mapsto \sigma  \st (\sigma,U,W) \in V\}
	\end{align*}
	The zone graph $Z(H)$ can be intuitively understood as a subgraph of $G$, where each zone vertex in $V_{Z(H)}$ is either a $(\sigma_i,U_i)$ of $H$, which is a contraction of the vertices $U_i$ of $G$, or a $(\phi_G(x),\{x\})$, which stems from $x$ being a neighbor of some vertex in $V_i$.
	
	For convenience, define $Y(H)$ as the subgraph of $Z(H)$ induced by H.
	%TODO: maybe write more about induction
	%TODO: comment about information duplication of labels
\end{definition}

\begin{definition}
	Let $h$ be a zone vertex, $r$ a production rule and $X$ a (potentially empty) set of parsing trees, $\ptree{h}{r}{X}$ is a parsing tree, whereby $h$ is called the root node and $X$ the children and $r$ is optional. $D(pt)$ gives a derivation for the parsing tree $pt$, which can be calculated by performing a depth-first walk on $pt$, starting from its root node, producing as result a sequence of derivation steps that correspond to each visited node and its respective rule. Additionally, a set of parsing trees is called a parsing forest.
	%TODO: Write D(pt) more formally. Explain the assembly of the effective derivation out of the sequence of sub derivation steps of the parsing tree nodes and explain how the isomorphims looks like. Because at the end, the derivation goes to a G', with G' in [G]
\end{definition}

Finally, the Algorithm \ref{alg:parse} displays the parsing algorithm of graphs with a NP BNCE graph grammar. Informally, the procedure follows a bottom-up strategy that tries to find production rules in $GG$ that generate zone graphs of $G$ until it finds a rule that generates a zone graph containing all vertices of $G$ and finishes answering yes and returning a valid derivation for $G$ or it exhausts all the possibilities and finishes answering no.

\begin{algorithm}[!h]
	\caption{Parsing Algorithm for NP BNCE Graph Grammars}
	\begin{algorithmic}[!ht]
		\Require $GG \text{ is a valid NP BNCE graph grammar}$
		\Require $G \text{ is a valid graph over } \Delta$ \Comment{$G$ has terminal vertices only}
		\Function{$parse$}{$GG=(\Sigma, \Delta, S, P), G=(V_G,E_G,\phi_G)$}{$:Derivation$}
			\State $bup \gets \{(\phi_G(x),\{x\}) \st x \in V_G\}$ \Comment{start $bup$ with trivial zone vertices}
			\State $pf \gets \{ \ptree{b}{}{\emptyset} \st b \in bup \}$ \Comment{initialize parsing forest}
			\Repeat
				\State $h \gets \text{\Select } \{X \subseteq bup \st\text{for all } U_i, U_j \in X \text{ with } i \neq j \. U_i \cap U_j = \emptyset \}$
				\ForAll{$d \in \Gamma$} \Comment{for each non-terminal symbol}
					\State $r \gets \text{any } \{(d \pro R,\omega) \in P \st R \isomorph Y(h) \}$
					\State $l \gets (d,V(h))$
					\If{$Z(\{l\}) \deriv{r}{l}{} Z(h)$}
						\State $bup \gets bup \cup \{l\}$ \Comment{new zone vertex found}
						\State $pf \gets pf \cup \{ \ptree{l}{r}{\{\ptree{z}{y}{X} \st \ptree{z}{y}{X} \in pf, z \in h \}} \}$
					\EndIf
				\EndFor
			\Until{$(S, V_G) \in bup$} \Comment{if found the root, stop}
			\State \Return $(S, V_G) \in bup \? \Just D(\ptree{(S,V_G)}{y}{X} \in pf) \: \Nothing $
		\EndFunction
		\Ensure $return \text{ is either } \Nothing \text{ or of the form } \Just \startG{GG} \derivtr{} G$
	\end{algorithmic}
	\label{alg:parse}
\end{algorithm}

The variable $bup$ ($bup$ stands for bottom-up parsing set, see ())%TODO: cite
is started with the trivial zone vertices of $G$, each containing only one vertex of $V_G$, and grows iteratively with bigger zone vertices that can be inferred using the grammar's rules and the elements of $bup$.

The variable $h$ stands for handle and is any subset from $bup$ chosen to be evaluated for the search of new zone vertices to insert in $bup$. The procedure $\Select$ gives one yet not chosen handle or an empty set and cares for the termination of the execution. Then, for the chosen $h$, rules $r$ with left-hand side $d$ and right-hand side isomorphic to $Y(h)$ that produce $Z(h)$ from $Z(\{l\})$ are searched. If any is found, then $l = (d,V(h))$ is inserted into $bup$. This basically means that it found a zone vertex that encompasses the vertices $V(h)$ (a possibly bigger subset than other elements in $bup$), from which, through the application of a sequence of rules, we can produce the subgraph of G induced by $V(h)$. This information is saved in the parsing forest $pf$ in form of a parsing tree with node $l$ and children $\ptree{z}{y}{X}$, already in the parsing forest $pf$, for all $z \in h$.

If, in some iteration the zone vertex $(S, V_G)$ is inferred, then it means that the whole graph $G$ can be produced through the application of a derivation starting from the start graph $\startG{GG}$ and thus $G \in L(GG)$. This derivation is, namely, the result of a depth-first walk in the parsing tree whose root is $(S, V_G)$. If, otherwise, all possibilities for $h$ were exhausted without inferring such zone vertex, then $\Nothing$ is returned, what means that $G$ cannot de parsed with $GG$ and therefore $G \notin L(GG)$.

%TODO: Talk about the empty productions, that are allowed
%TODO: Talk about the isomorphism in the reduction test

%TODO: We beleieve this imperative view is also a contribution of our work

%TODO: Comment on correctness and completeness
%TODO: Comment on termination and spatial and time complexity
%TODO: Nevertheless, the grammar's ambiguity does not compromise the correctness of our algorithms presented in the following.